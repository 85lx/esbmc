\documentclass{article}
\usepackage{url}
\usepackage{listings}
\usepackage{tabularx}
\usepackage{longtable}
\usepackage{titling}
\newcommand{\subtitle}[1]{%
  \posttitle{%
    \par\end{center}
    \begin{center}\large#1\end{center}
    \vskip0.5em}%
}

\begin{document}
\author{Jeremy Morse}
\title{A manual on some internals of ESBMC}
\subtitle{Caveat lector}
\maketitle
\tableofcontents
\section{Introduction and caveats}

This ``manual'' is supposed to be an introduction to how the ESBMC model
checkers internals are arranged and operate. It is not supposed to be a
comprehensive piece of documentation on the exact behaviour of particular
functions or facilities; nor will it ever, ever be up to date. Exact
documentation on a particular function or method should be written in
doxygen in headers; this documentation can be built to HTML by executing
\texttt{make doxygen} in the top level directory of ESBMCs source tree.

When referencing portions of code from within this manual, I'll probably
end up referring to class names and methods within them. Source files and
line numbers are liable to change, wheras the code layout of the project
is the least likely to suffer significant churn. The location of such a
class or method should be obvious from the context, or discoverable with
grep. Numerous references will also be made to the \textit{Internal
representation}, or \textit{irep} of something. This refers to how some piece
of data is structured or stored, see the 'Misc' section for detail.

A huge amount of the code base is derived from the CBMC project. CBMC is
open source (BSD 4-clause, ish), and available over SVN at
\url{http://www.cprover.org/svn/cbmc}. A large number of design decisions
are down to the development of CBMC; changes to ESBMC that cause
significant divergance from CBMCs design should be carefully thought
about, seeing how it's more mature than ESBMC. Likewise, code being
pulled in from CBMC should be examined to see whether it'll actually fit
into what ESBMC is doing nowdays.

All additional gunge, queries, complaints, to \url{jeremy.morse@gmail.com}

\section{Source tree structure}

In an order vaguely related to how ESBMCs execution order occurs, the
following describes the contents of directories in the source tree.

FIXME: This is ugly, make it a table.

\begin{description}
\item[docs] Directory for not-in-code documentation.
\item[papers] Self explanatory.
\item[scripts] Various scripts and auxilary files related to building ESBMC
               and dealing with things that aren't source files. Makefile
               scripts and release/binary manipulating scripts.
\item[esbmc] Top level model checking control code. Process entry point,
             option handling, general direction and invocation of the rest
             of the code base.
\item[langapi] Abstractions for handling input source files. Links a variety
               of global functions up to input-language-appropriate routines.
               Probably not massively necessary and could be ditched.
\item[ansi-c] Parser for ANSI-C software. Contains all code required to lex,
              parse, store as an AST, typecheck, and link, an input file.
\item[ansi-c/cpp] C preprocessor - an import of the Portable C Compilers
                  preprocessor, adapted to do what ESBMC needs.
\item[ansi-c/headers] C langauge headers to override system headers.
\item[ansi-c/library] C language implementations of various code libraries
                      that we seek to model.
\item[cpp] Parser for C++ language. Code for all compilation steps of C++.
\item[cpp/library] Implementation / models of various C++ template libraries.
\item[big-int] Arbitary length integer library. Used internally to avoid any
               kind of problems modelling large machine integers using small
               machine integers.
\item[goto-programs] Routines for general operations on GOTO instructions, as
                     well as all the code for converting a parsed AST into
                     GOTO instructions.
\item[pointer-analysis] Code for interpreting the execution of GOTO instructions
                        and the analysis of their effect upon pointer tracking.
                        Basically a static analysis of pointer assignment and
                        reachability. Also, contains code for resolving pointer
                        indirection in dereferences.
\item[goto-symex] Symbolic execution of GOTO instructions into an SSA program.
\item[solvers] Encoding of SSA program into SMT solver logic, and solving of
               the produced SMT formula.
\item[util] Miscellaneous functions, classes, and whatever to glue everything
            else together.
\item[regression] Regression tests for various different facets of ESBMC.
\end{description}

\section{Command line options}

What follows is a short description of each command line option to ESBMC that
exists. All are prefixed with \url{--} when given. Some information is also
available by running \url{esbmc --help}.

\begin{longtable}{| p{.20\textwidth} | p{.80\textwidth} |} 
\hline
Option & Description\\
\hline
inlining & Enables inlining of functions -- function calls are replaced with the
bodies of the called function. It's unclear when these occur.\\
\hline
program-only & Symbolically execute the program until we have an SSA program,
print the SSA program textually, then exit.\\
\hline
program-too & Like program-only, but don't exit, instead continue
verification.\\
\hline
function & Specify function to begin execution at. If function takes arguments,
horrible undefined behaviour occurs.\\
\hline
preprocess & Run the input source files through preprocessing, write the output
to stdout, then exit. Useful for debugging.\\
\hline
no-simplify & Disable simplification of expressions; effectively neuters
constant propagation so that no fact is statically determined to be true or
false, and we end up always exploring to the top of the unwind bound.\\
\hline
unwind & Takes integer parameter. Specifies the general unwind bound to apply
to all loops in the program.\\
\hline
unwindset & Takes string parameter, a comma seperate list of loopidnum:bound
pairs. Allows exact specification of a loop bound for a particular loop, see
the show-loops option.\\
\hline
z3-bv & Use Z3 solver in bitvector mode.\\
\hline
z3-ir & Use Z3 solver in integer mode.\\
\hline
boolector-bv & Use Boolector solver.\\
\hline
outfile & Output file to write an SMT formula to.\\
\hline
no-pointer-check & Disable all assertions related to pointers.\\
\hline
depth & Takes parameter n. Limit the maximum depth of the program to the
execution of n instructions.\\
\hline
no-div-by-zero-check & Disable divide-by-zero assertion checks.\\
\hline
no-unwinding-assertions & Disable unwinding assertions.\\
\hline
partial-loops & Disable unwinding assumptions.\\
\hline
memory-leak-check & Enable checking for memory leaks.\\
\hline
overflow-check & Enable integer overflow assertions.\\
\hline
no-assertions & Disable the checking of any assertion encoded in the program
with an \url{assert} function call.\\
\hline
minisat & Use the minisat sat solver.\\
\hline
16 & Model a 16 bit machine. Unlikely to work as this hasn't received any
maintenence, ever.\\
\hline
32 & Model a 32 bit machine. The default.\\
\hline
64 & Model a 64 bit machine. Will probably have hiccups involving pointer
widths.\\
\hline
little-endian & Model integers as being little endian. The default.\\
\hline
big-endian & Model integers as being big endian.\\
\hline
show-goto-functions & Print out all the functions and instructions in all the
input code.\\
\hline
show-value-sets & Print out the same data as show-goto-functions, but also
add the contents of the pointer tracking set (according to the static analysis)
at each instruction.\\
\hline
show-loops & Print out a list of loops, where they are in the source files,
and what loop ID number they have.\\
\hline
show-symbol-table & Print out a list of symbols and some simplified
information about them.\\
\hline
show-claims & Print a list of all assertions in the program.\\
\hline
claim & Takes integer argument identifying what assertion to check in the
program, from the list in show-claims. Disables all other assertions. Specify
the claim option multiple times to check multiple claims.\\ 
\hline
atomicity-check & During multithreaded exploration, verify that expressions
are executed atomically. Rewrites assignments to check that the variables on the
rhs are not modified between before the assignment and afterwards.\\
\hline
error-label & Check for reachability of a particular C label in the program.\\
\hline
version & Print the version of ESBMC.\\
\hline
no-lock-check & Don't make assertions about correct operation of pthread
mutexes.\\
\hline
deadlock-check & Encode assertions checking that the program under test does not
deadlock.\\
\hline
string-abstraction & Enable an approximation of string operations. All string
operations only consider the lenght of data, rather than its contents.\\
\hline
no-slice & Disable slicing step.\\
\hline
qf\_aufbv & Print the SMT formula for the program to the file given in the
outfile option, in QF\_AUFBV logic.\\
\hline
qf\_auflira & Print the SMT formula for the program to the file given in the
outfile option, in QF\_AUFLIRA logic.\\
\hline
context-switch & Takes one integer parameter. Specifies the number of thread
context switches allowed during exploration. Defaults to infinite.\\
\hline
time-slice & How many time slice to permit during round-robin scheduling.\\
\hline
k-step & How many k-unrolls to make during k-induction.\\
\hline
no-por & Disable partial order reduction in multithreaded checking.\\
\hline
data-races-check & Check to see whether any concurrent reads/writes to a
variable are possible. Rewrites assignments to encode an assertion that fails
if an interleaving is found where either two threads can write to the same
variable at the ~same time, or can write while the other reads.\\
\hline
DFS & Perform depth first search scheduling of multithreaded code. The
default.\\
\hline
schedule & Perform a ``schedule'' scheduling of multithreaded code, encoding
all interleavings into one SSA program.\\
\hline
all-runs & Explore all interleavings in the program, even after a counterexample
is found.\\
\hline
timeout & Start a SIGALRM to time out and kill ESBMC in the future. See --help
for more details.\\
\hline
memlimit & Encode a resource limit to kill ESBMC if it uses too much memory.
See --help.\\
\hline
state-hashing & Enable state hashing. Attempts to hash program state in
multithreaded exploration and reduce duplicate states.\\
\hline
symex-trace & Print GOTO instructions as they are executed. Useful for
debugging.\\
\hline
round-robin & Enable round robin thread scheduling.\\
\hline
k-induction & Enable k-induction.\\
\hline
break-at & Takes integer parameter. Executes an x86 trap instruction when the
specified GOTO instruction is executed. This is the equivalent of putting a
breakpoint on it.\\
\hline
memstats & Cat /proc/self/status at the end of execution, to give details on
how many bytes of memory were used, and so forth.\\
\hline
dump-z3-assigns & Print all assignments to SMT symbols if a counterexample
is found.\\
\hline
symex-ssa-trace & Print SSA instructions as they are encoded. For debugging.\\
\hline
\end{longtable}

\section{Top level procedures}

Entry to the process starts (more or less) in the \url{doit} method of the
\url{cbmc_parseoptionst} object. Various command line options are checked
for validity, before the \url{get_goto_program} method invokes the
frontend parsers to compile input source code into an AST. The AST for the
entire environment (all source files and libraries) is stored in a
\url{contextt} object, containing a list of symbols and their AST value.

The contents of the \url{contextt} object is passed to the
\url{goto_convert} function, which produces a set of
\url{goto_functiont}s corresponding to each function in the source language.
Each function contains little more than a \url{goto_programt}, which
actually contain a list of instructions and some annotations.

With the set of GOTO functions, the \url{process_goto_program} method
applies the string abstraction transformation, the pointer analysis,
installs various pointer validity checks, and anything else that transforms
the source program into different instructions (such as LTL property monitors
or data race checking).

With these fixed-up goto functions, a \url{bmct} object is created and the
\url{run} method invoked on the functions. These functions are fed into a
\url{reachability_treet} object, the primary interface to symbolic
execution. Within the \url{bmct::run} method, the symbolic execution engine
is asked to run through instructions creating an SSA program; potentially
several times if there are multiple threads involved. A result itself is a
\url{goto_symext::symex_resultt} containing the SSA program container and
a count of how many assertions remain to be verified in the program.

The SSA program is then optionally sliced; see the 'Misc' section for details.

A solver object is then created, a subclass of \url{bmct::solver_base}
abstraction which solver to use. The SSA program is then fed to the solver,
which encodes it to SMT or whatever appropritate encoding it uses. It's then
asked to solve the equation, returning:
\begin{description}
\item[UNSATISFIABLE] The formula isn't satisfiable.
\item[SATISFIABLE] The formula is satisfiable.
\item[SMTLIB] A special case for printing the formula to a SMTLIB file.
\item[ERROR] Some error occured during solving.
\end{description}

Finally, if the formula is satisfiable, an error trace is created and printed.
Further details in the 'Misc' section.

\section{Source file parsing}

Code parsing is one of the untouched pastures of CBMC code, mostly. The ANSI-C
frontend is almost entirely like the original, while the C++ frontend has been
significantly developed by Manaus. The author is really familar with neither.
I'll talk about the ANSI-C frontend, then how the C++ frontend relates to it.

There are some significant conceptual steps involved. Firstly consider
the input and the output. Coming in is a C source file that must be preprocessed
and parsed - two fairly straightforward (although not easy) tasks. The top
levels of ESBMC then receive an AST representing the types and code structure
of the source file, which is more complex. The irep / structure of this data
is entirely undocumented and closely coupled between the parsers and the
\url{goto-programs} dir that converts it to GOTO instructions. However the
contents of this AST tends to be fairly high level language constructs,
for example \url{for} and \url{switch} statements. Refer to the method
\url{goto_convertt:convert} for an idea of what kind of constructs these
are. The majority of the source file parsing code deals with converting between
the parse tree and the AST.

The preprocessing stage is contained in the \url{c_preprocess} function.
In CBMC this used to offload preprocessing to the host preprocessor, however
our requirements have become more complicated since then. We now do
preprocessing using the preprocessor from the Portable C Compiler project
(which is BSD licensed). Unfortunately it wasn't designed with memory management
in mind, so ESBMC picks an output file, forks, calls the preprocessor to pump
outupt to the selected file, then exits the child process.

The complicated requirements from the preprocessor is that we generally want
code under test to have access to all headers on the host system, however we
also want to shoehorn our own types and functions in there --- for example the
glibc headers for assert have some obnoxious defines involved. Additionally
given how much lee-way standards give to libraries to define the format of
opaque data structures, we may need to define our own data structures to
avoid having to special case code for different operating systems. For example,
\url{pthread_t}'s, \url{pthread_mutex_lock}'s and so forth differ
between operating systems, and we currently rely on on using Linux' pthread.h
header. The current fix for this is to intercept \url{\#include} statements
and read in an ESBMC specific header file from \url{ansi-c/headers} rather
than the system headers.

We parse C in the normal way; a flex tokenizer is defined, and a yacc grammar
defined which translate the input C into a parse tree. This needs no special
description. A class (\url{ansi_c_convertt}) takes the parse tree and
makes a simple translation to the format of the AST. The bulk of the work then
lies in the \textit{typechecking} phase. Here, the nasty parts of C that are
context-dependant\footnote{i.e., all of them} are fixed up. Factors such as
integer promotion, operation signedness, and actual correctness are
considered, and various casts or extensions are inserted. The output is stored
as a set of symbols with associated AST values in a \url{contextt} object.

After parsing and compilation is linking. In the past CBMC has just
tacked all libraries available onto the end of a source file being compiled,
and that's all. Nowdays the libraries are pre-compiled into a binary
representation of GOTO instructions, and linked in after typechecking of C
code, by copying in any symbol referred to from the compiled source files
that are in the compiled library files.

Finally there's the initialization of C global and static lifetime variables.
Seeing how the GOTO language is only made up of assignments, ish, their
initialization must be made by assignments too. So, an initial 'main' GOTO
function is synthesized from the \url{c_main} and
\url{static_lifetime_init} functions. For each global variable an
assignment is emitted assigning the initial value to the global variable.

\section{GOTO programs}

This section covers both the GOTO program record itself, and the GOTO
instructions that make it up. Before launching into a description of these
records, it is important to understand that CBMC synthesizes the
\url{goto_programt} class from the \url{goto_program_templatet}
template. This can lead to the most confusing and obscure error messages
if you do not realise that you're manipulating a template. From a design
point of view, the reason for this appears to be so that the types of
the GOTO program body could be parameterised; a decision that is almost
entirely without merit.

The \url{goto_programt} class is more or less just a container for a
list of GOTO instructions (of class \url{goto_programt::instructiont}).
It stores \textbf{no} additional information. Instead, all of its methods
perform operations on the contained instructions. Most of these relate
to the creation, insertion, and deletion of instructions, recalculating
their contents to be consistent after such a modification, and a few
special cases such as determining the successor instructions from a
particular instruction,

The actual GOTO instruction class itself's primary piece of data comes in
two flavours -- the \url{code} member or the \url{guard} member.
These store the internal representation of what the body of the instruction
\textit{is}. Exactly what the instruction means depends on the \url{type}
field, described thus:

\begin{description}
\item[GOTO] Jump from the current instruction to the instruction in the
\url{targets} field. If \url{guard} is not true, then the jump is
conditional, depending on the evaluation of \url{guard}. If \url{guard}
is true the jump to the target occurs; if not, execution continues to the next
instruction.
\item[ASSUME] Encode an assumption, stored in the \url{guard} field, to the
solver.
\item[ASSERT] Encode an assertion, stored in the \url{guard} field, to the
solver.
\item[OTHER] Catch-all instruction for storing special cases, enumerated below.
Identified by what kind of irep is stored in the \url{code} field.
\begin{description}
\item[cpp\_delete] Also \url{cpp_delete[]}. Represents a deallocation of
some memory allocated by C++'s \url{new} or \url{new[]} operators.
\item[printf]\footnote{Yes, really} Represents a printf operation, for later
printing in a counterexample.
\item[decl] Represent declaration of a variable. Normally the declaration of
a variable is uninteresting as we only care about when it is initialized.
However in a loop where a variable is declared inside the loop block, it
transitions from being initialized to uninitialized when the loop iteration
finishes. Hence the importance of knowing where it is declared.
\item[nondet] Represent a nondeterministic value, from a \url{nondet_*}
function call.
\item[asm] Inline assembly statement. Mercifully ignored.
\item[typeid] Fetch a C++ type ID record, I belive.
\end{description}
\item[SKIP] An ignored instruction.
\item[LOCATION] Previously caused a ``location'' SSA step to be recorded for
future tracking of the code path of the counterexample. Now redundant.
\item[END\_FUNCTION] End instruction of a function. Not the same as a return,
which can occur anywhere, but actually the final instruction in the list of
instructions.
\item[ATOMIC\_BEGIN] Self explanatory.
\item[ATOMIC\_END] Self explanatory.
\item[RETURN] Record a return statement, identifying the expression to return.
Stored in a ``return'' irep in the \url{code} field.
\item[ASSIGN] Self explanatory. An ``assign'' irep is stored in the
\url{code} field, identifying the left and right hand sides.
\item[DECL] Unused. Probably used to be, or was intended to be, the decl
irep from the OTHER instruction.
\item[DEAD] Unused. Comments say ``marks the end-of-live of a local variable''.
\item[FUNCTION\_CALL] A function call record; stores a function call irep in
the \url{code} field, which in turn records the left hand side of the call,
the arguments, and the target.
\item[THROW] Throw record; not familiar with this, but it'll result in some
kind of an assignment to a record of what's been thrown, and a jump to somewhere
else.
\item[THROW\_DECL] Record the start of a catch block for a particular type
of variable.
\item[THROW\_DECL] Record the end of a catch block for a particular type
of variable.
\end{description}

All behaviours of GOTO programs are described by lists of these instructions.
Additional annotations are stored with each instruction, for example the
\url{function} and \url{location} fields identify where in the source
files the instruction came from. The \url{targets} list contains a list
of where the instruction can jump to (which should only ever contain zero or
one target instructions).

\url{loop_number} identifies a unique loop number for backwards
GOTO instructions. \url{target_number} is a numeric ID that labels
the instruction within a function. This don't actually do anything, but
is printed in the textual representation of GOTO instructions to indicate
the targets of GOTOs.  There's also a set of local variable names, and
a globally unique instruction ID in \url{location_number}.

That's the substance of instructions; more information on the interpretation
of them lies in the symbolic execution section.

\section{Pointer analysis}

The essence of the pointer analysis is a tracking of what pointer variables
exist in the GOTO code, and what they might point at. This occurs more than once
during each run of ESBMC. A static analysis of the instructions first attempts
to establish a set of all (lexical) variables that a particular (lexical)
variable in a function may point at. Then during symbolic execution, a similar
tracking maintains a set of (``runtime'') variables that an actual pointer
\textit{does} point at.

The static analysis is initiated from the GOTO program processing code in the
\url{cbmc_parseoptionst} object. The high level analysis logic actually
lies in the \url{goto-programs} directory with the
\url{static_analysist} and \url{abstract_domain_baset}
classes. Code in these classes call abstract methods to perform transformations 
between states as appropriate, over all GOTO instructions, to find a fixedpoint
where all values of the abstract domain have been discovered for all states.

The \url{value_set_analysist} and \url{value_set_domaint} classes
subclass the above two classes respectively to provide concrete
methods\footnote{I'm probably using all the wrong terminology here} for
tracking states of what pointer variables might point at. Most of the logic
itself lies in the latter class, storing both the actually tracking data
and forwarding transformation method calls to the appropriate objects.

A \url{value_set_domaint} contains only a \url{value_sett} object.
That itself contains the pointer tracking map, which is, unsuprisingly,
string based. The core type is the \url{value_sett::valuest} map,
where a string identifying a variable name maps to a
\url{value_sett::entryt}, which stores a set of variables that may be
pointed at and the offset into them.

The string key of each of these entries is important -- When interpreting 
an assignment of a pointer value to a variable, we take the original variable
name being worked on and then interpret the left hand side, appending strings
to indicate /what/ part of the variable is being assigned to. To illustrate,
consider an assignment to the \url{bees} field of the following struct:
\begin{lstlisting}
struct face {
  void *bees[4];
};

int main() {
  struct face knees;
  knees.bees[0] = NULL;
  return 0;
}
\end{lstlisting}

Here, the fully qualified name of the variable we are assigning to is
\url{main::main::knees}, which becomes the starting point for the string
key in the value tracking map. We then interpret the left hand side of the
assignment, observe that we access the \url{bees} field, and so append the
text \url{.bees} to the key we are calculating. The next part of the left
hand side is the access to an element of the \url{bees} array, so we
append the text \url{[]} to the key we calculate. The final key is then
\url{main::main::knees.bees[]}. Observe that this approach allows every
variable in the program to have a unique key in the tracking map, except for
elements in an array --- we instead track what \textit{all} elements of the
array may point at, thus forming an overapproximation. The reasons for this
should be obvious.

The \url{value_sett::entryt} class is responsible for tracking the target
variables that a pointer may point at. It stores a map between certain variable
names and \url{value_sett::objectt}s. The presence of a variable name key
in the map indicates that the variable may be pointed at. The
\url{value_sett::objectt} object records whether the offset into the
variable that is pointed at is nondeterministic or constant; and if fixed,
then what the offset is. (NB: the actual implementation of this stores
\url{symbol} ireps as the map keys. To optimise this, it uses a (global)
pooling technique to assign each irep an ID number; see the
\url{value_sett::object_numbering} object. The ID number is then used
as the key into the \url{value_sett::entryt} map).

The \url{value_sett} class also provides operations required in the course
of the static analysis, most importantly the ability to interpret an instruction
to record and update the tracking described above. The class can also merge
value set records into each other. This follows the obvious merging procedure;
however when the two tracking sets being merged have a pointer variable
that points at different offsets into the same data object, the merged
tracking set records a nondeterministic offset into that data object. This
forms an overapproximation of the offset into an object that a pointer points
at.

No attempt is made to track what I'll term \textit{funky} pointer assignments.
For example, if code deconstructs a pointer variable into bytes, then
reconstructs these bytes into a pointer value, we are unable to track
what the resulting pointer value points at. How to address this in the future
is an open question. The byte array memory models of other tools neatly
side-step this issue.

The static analysis process eventually reaches a fixedpoint state where we
have established all possible variables that may be pointed at byte pointer
variables. The contents of this analysis is then handed to an object of
class \url{goto_program_dereference}. This proceeds to enumerate all
GOTO instructions and attempts to perform all dereferences in the instruction.
Pointer safety assertions are then generated (see the section on dereferencing)
and inserted as ASSERT instructions prior to the dereferencing instruction.

The pointer analysis executed during symbolic execution uses the same records
and functions as the static analysis. While the static analysis attempts to find
all the variables a pointer might point at across all code paths, the symex
tracking instead tracks the set of all variables a pointer may point at in the
course of the current code path. The variables it tracks are also ``L1 renamed''
(see the section on Symbolic Execution).

It is speculated that the static analysis can be removed, and assertions
encoded on-the-fly when dereferences occur during symbolic execution. While
this might be a valid optimisation, the TACAS13 performance figures indicate
that execution time is dominated by symex, rather than the pointer
analysis\footnote{660 seconds ``GOTO processing'' compared to 20,000 seconds
``BMC time''}.

\section{Symbolic execution}

This portion of ESBMC is likely the most important in terms of theory,
complexity, and performance. The overall task is to take an input set of
GOTO functions, and interpret their instructions through an execution path
that is bounded according to the rules of BMC. Along the way, a \textit{Single
Static Assignment} (SSA) program is created recording the operations that
occured during the trace, for later checking. We must also make decisions
regarding the exploration of thread interleavings if multiple threads are
involved. To cover this section, we'll first look at the substance of the
SSA program, the concept of variable ``renaming'', the classes that make up
the symex process, how specific instructions are interpreted in single
threaded code including certain special case library functions,
and finally how all this relates to multi-threaded code.

\subsection{SSA programs}

The output from a single symex run through a program is contained in a
\url{symex_target_equationt} object. This primarily stores a list of
\url{symex_target_equationt::SSA_stept} objects, each of which represents
a single operation in program, sometimes referred to as a \textit{trace}.
The idea of SSA \textit{variables} will be discussed in the ``Variable
Renaming'' section. A SSA step comes in four flavours:

\begin{description}
\item[ASSIGNMENT] An assignment to a variable, with a symbol on the left hand
side, and an expression irep on the right hand side.
\item[ASSERT] Represent an assertion that an expression evaluates to true.
\item[ASSUME] Represent an assumption that an expression evaluates to true.
\item[OUTPUT] Record an output from the trace. This is essentially a wrapper
around a printf operation, that causes variables to be printed in the
course of printing a counterexample. I recommend not asking.
\end{description}

Numerous expression fields in a step object record the details of the above
operations, and are generally uninteresting for this discussion. The only
other data of note is that each step stores a \url{source} object, recording
where in the GOTO program the SSA step was generated.

\subsection{Variable renaming}

FIXME: this section is likely worded in a cack handed manner, requires
verification.

In the course of this manual there are numerous references to ``variables'',
without futher elaboration. The different aspects of variables that might be
referred to are below:

\begin{itemize}
\item The lexical variable, i.e. the variable name itself in a particular
context, not specific to a particular value or function activation.
\item The storage of a variable. The portion of memory reserved for storing
the value of a variable, either a global variable or memory allocated on the
stack if an automatically allocated variable in a function block.
\item The contents of a variable. Simply the value assigned to the variable
at a particular point in a program.
\end{itemize}

We name these different aspects \textit{level 0}, \textit{level 1}, and
\textit{level 2} respectively, often shortened to L0, L1, or L2. The reason
for this naming will become obvious shortly. An example is in order to fully
understand this. Consider the following function, and how it relates to the
pointer analysis:

\lstset{numbers=left}
\begin{lstlisting}
int anint;

void somefunc(int **beans) {
  int *bears = *beans;
  bears = &anint;
}

int main1(void) {
  int wololololo = 0;
  somefunc(&wololololo);
}

int main2(void) {
  int ponies = 0;
  somefunc(&ponies);
}
\end{lstlisting}

Here, if we consider the variable \url{beans} as an L0 variable, what it
points at is the set of all pointer targets it could potentially point at over
\textit{all} execution traces that are available in the program, including
all pointers that may be fed into \url{somefunc} at any point in the
program. This set includes the \url{wololololo} and \url{ponies}
variables. However if we consider \url{beans} as an L1 variable, what it
points at is whatever it may point at across the lifetime of its storage. Thus,
this depends on the path taken through the program -- the pointer may point
at \url{wololololo} or \url{anint} if \url{somefunc} is called from
\url{main1}, or alternately it may point at \url{ponies} or \url{anint}
if called from \url{main2}. However it may never potentially point at all
three. Finally, L2 variables are the actual value of the variable at a
particular point in the program, in our example the L2 \url{beans} pointer
may only ever point at a single value, which varies with both the path through
the program and the actual instruction location.

(These differences are important during the pointer analysis -- the static
analysis tracks pointers as L0 variables, the symex pointer tracking considers
L1 variables).

The reason for describing these variable kinds in terms of ``level''s, is
that we can consider a higher ``level'' of variable as representing a set of
lower level variables. Consider: if we call the \url{somefunc} function
in the above example three times, then the L0 \url{beans} variable exists
only once (as the function only exists once, so the lexical variable exists
once). However, three L1 instances of the L1 \url{beans} variable exist,
because the function was called three times and storage allocated for the
variable three times. In the same way, repeated assignment to a variable
in a function causes only one L1 name to exist (as only one piece of storage
is required), however multiple L2 variables exist as the variable has
multiple values during the execution of the function.

This then leads to the concept of ``renaming'', which is where we take a high
level variable and reference a lower level variable, appropriate to the context.
The two forms this can take are the transitions L0 to L1, and L1 to L2. In the
former case, we are taking a lexical variable and referring to the actual
storage of that variable. If the variable is global or of a static lifetime,
no additional data is added to the record, as the variable is always the
same piece of storage across the whole program execution. If the variable is
allocated local to a function call, then the L0 variable is annotated with
a unique \textit{activation record} number that identifies which invocation
of the function we are dealing with, and a thread number. The actual values
of these numbers are derived from the context of the renaming -- a thread number
is always available, and the activation record number of the function is always
available (as it's illegal to refer to a local variable outside of its
scope\footnote{Indirect pointer references are special, see later}).


When renaming an L1 variable to an L2 variable (effectively finding the value
currently stored by that L1 variable) we annotate the L1 variable with
a context switch ID number (see later sections) and an \textit{SSA assignment
number}. The assignment number is a monotonically increasing counter giving each
assignment to an L1 variable a unique identifier. So, if we assign to a
variable four times in a function, four L2 variables are created with
assignment numbers from one to four. This preserves the SSA constraint that we
only ever assign to a variable once. When renaming an L1 variable to L2 for the
purpose of using its value rather than making an assignment, we take the
assignment number to be the greatest assignment number for that L1 variable,
thus giving us the most recently assigned value of that L1 variable.

The final complexity to this situation, is that there is never a particular
object in ESBMC that represents any of these variables. Instead, variables
are identified by \textit{name}, stored in a symbol irep. Unsuprisingly, this
name is a string\footnote{Changes in the irep2 branch}. Multiple symbol irep
objects can contain the name of the same variable; creating a new variable
at any leve is as simple as creating a new name. An interesting side-effect
of this is that, following the procedure for renaming an L1 variable to L2,
if the L1 variable has never been assigned to then the greatest assignment
number is zero, and so references to unassigned variables read from assignment
number zero. (And because when we reach the solver level there is nothing
constraining the zeroth assignment of that variable, it's a free variable).
The format of this string is as follows:

\begin{quote}
full\_variable\_name@actv\_record!thread\_no\&cswitch\_no\#assign\_no
\end{quote}

Where \url{actv_record}, \url{thread_no}, \url{cswitch_no}, and
\url{assign_no} are replaced with the appropriate numbers. L0 names only
feature the variable name at the start; L1 names follow the above format
up until the ampersand; and L2 names use the full format. These names will
frequently crop up throughout almost all of ESBMC.

\subsection{Class overview}

The top level class for symex is \url{reachability_treet}. This contains,
at some level, all the state involved in the interpretation of GOTO
instructions. It also has the high level entry-to-symex methods like
\url{reachability_treet::get_next_formula}. Almost all the actual
logic in the class is related to the exploration of multithreaded interleavings,
everything symex related exists in lower level classes. The main piece of
data stored is a set of \url{execution_statet}'s, each of which stores
the full state of the program at a particular point in time.

The next two classes of interest are the \url{goto_symext} and its
subclass, \url{execution_statet}. The former contains all the logic
for GOTO interpretation and anything else only related to single threaded
execution, plus stores a few parameters for symex exploration. The latter
overrides a number of methods and injects logic for discovering when
multithreading operations must occur. The logic in this class only relates
to operations on the program state rather than interleaving exploration.
It stores a set of thread state objects, of class
\url{goto_symex_statet}, while some specialised information is broken out
into \url{execution_statet} for easy accessibility (such as what threads
are still running, atomic blocks, startup parameters). Global program state
is also stored here as a \url{value_sett} object to track pointer value
sets, and a \url{renaming::level2t} storing data on the latest assignments
to variables.

Within the \url{goto_symex_statet} class are all thread specific pieces
of data --- things like the program counter (in
\url{symex_targett::sourcet}) or the call stack (in
\url{goto_symex_statet::framet}). Significantly more state is stored
tracking the nondeterministic exploration of the program.
\url{goto_symex_statet::goto_statet} objects contain the whole program
state resulting from a short deterministic path, which are then merged together
where control paths merge\footnote{i.e., SSA phi functions} to form the
nondeterministic program trace. Various objects of \url{guardt} class
record the guard of a particular path being taken. Also stored is the count
of how many times a loop has been unwound in the current context, for the
purpose of loop bounding.

A \url{goto_symex_statet::framet} stores information related to a
particular stack frame / function activation. Mundane facts like the function
name, call site location, return value variable and the names of local variables
live here. More exciting items such as the set of executions to be merged in
the future, and the actvation record number for renaming, live here too.
There are also some hacks related to function pointer interpretation.

\subsection{Instruction interpretation}

The execution of instructions begins in the constructor for
\url{execution_statet} where the first thread is created and its program
counter set to the entry of the \url{main} function. From that point
onwards, the \url{goto_symext::symex_step} method is repeatedly called
to take the next instruction, interpret it, and move forwards. Some
instructions are handled in \url{execution_statet::symex_step},
which overrides its superclass and interprets multithreading specific
instructions. What follows is a narative of the (not quite) high level
operations performed during the interpretation of these instructions. Assume
that pointer dereferences are already handled as described in the section
on dereferencing --- actions specific to dereferencing will be mentioned
explicitly.

\subsubsection{Assign}

The substance of an assignment is a left hand side and right hand side. The
outcome we need is an SSA assignment, where we calculate a value with a type
n the right hand side, and then create a new left hand side variable that this
right hand side can be bound to. Clearly the assignment of a constant rhs to
a variable on the lhs is the simplest example of this. The operation becomes
more complex when the left hand side is an array index, a struct field,
a cast, or a nondeterminstic symbol due to pointer dereferencing. These
assignments must be rewritten into a simple variable assignment.

Performing this rewriting for an array index, we take the array variable
on the left hand side, and create a WITH operation that updates the desired
element with the right hand side. The assignment is rewritten to become an
assignment of this new value to the array variable. A similar operation
occurs for struct and union member assignments. For a nondeterministic
assignment we encode multiple deterministic assignments, one for each left
hand side, guarded by the appropriate guard. For assignment to a
\url{byte_extract} irep, we replace the right hand side with a
\url{byte_update} irep of the base data object on the left hand side.
The logic for all these operations begins in the
\url{goto_symext::symex_assign_rec} method and methods called from there.

Once we have an assignment in this form it is suitable for being an SSA
assignment. However all the variables are L0 variables, and need to be
renamed to refer to the appropriate piece of data or value. In the course
of the \url{goto_symext::symex_assign_symbol} method we call
\url{goto_symex_statet::rename} which identifies each variable on the
right hand side, renames to L1 (storage variables), and then to L2 (actual
values). The identifying numbers for these operations are pulled from the
top level \url{goto_symex_statet::framet} object of the most recently
called function, and the global L2 state tracking. One notable exception here
is that any operand to an \url{address_of} operation will only be renamed
to L1 --- we do not take the address of a particular variable value, but
instead the storage of the variable.

Once renamed, execution passes to \url{goto_symex_statet::assignment},
which performs accounting for the assignment on the L2 level. Specifically,
it bumps the SSA assignment number for the variable assigned to, so that future
uses of it will have the updated / recently assigned value. Additionally
we inspect the contents of the right hand side, and if it is a
sufficiently\textsuperscript{\texttrademark} constant value it is cached
in the program L2 state tracking, for constant propagation. Future rename
operations on that variable will have the variable replaced with its constant
value, rather than the variable itself. This method also passes the
assignment to be interpreted by the pointer analysis, which updates what the
left hand side may possibly point at.

Finally the \url{symex_target_equationt::assignment} method is called
to encode an SSA assignment, the program counter is incremented to point at the
next instruction, and we are done.

\subsubsection{Assert and assume}

ASSERT and ASSUME instructions are no-where near complex. Their substance
is the \url{guard} field of the GOTO instruction, which records the condition
that we are asserting or assuming. This expression is renamed as above to refer
to the appropriate L2 or L1 variables, and is passed to the expression
simplifier in case we can determine the truth of the operation statically.
The time and particularly memory benefits of determining this early are
significant.

The relevant methods are \url{goto_symext::claim} for ASSERT and
\url{goto_symext::assume} for ASSUME. These perform the above operations,
and then call \url{symex_target_equationt::assertion} and
\url{symex_target_equationt::assumption} respectively.

\subsubsection{Goto}

GOTO is likely the most complex of all the GOTO instructions, due to the sheer
amount of juggling of state that results. The substance of these is again
the \url{guard} field of the instruction, and a target instruction where the
GOTO is jumping to. There are two other factors that significant decisions
are made on --- firstly whether the GOTO is \textit{backwards} to an earlier
instruction in the program, and whether the guard of the instruction is (when
simplified) a constant true or false, or nondeterministic.

Consider the naive execution of some GOTO instructions. We start by executing
the first instruction in the program, moving on to the next, and repeating until
we reach the last instruction in the program. We can also handle loops by,
whenever we execute a GOTO that jumps backwards, following that jump and
executing from an earlier instruction in the program. This basic approach to
interpretation is what is employed by ESBMC, with all control flows more
complicated than the above encoded by making portions of the execution trace
nondeterministically executed.

The best way to describe this is through an example: below is a sample piece
of C, and an approximation of the corresponding GOTO code that it translates to.

\begin{lstlisting}
if (somebool) {
  a = b;
} else {
  a = c;
}
return a;
\end{lstlisting}

\begin{lstlisting}
   IF somebool GOTO 1
   ASSIGN a = b
   GOTO 2
1: ASSIGN a = c
2: RETURN a
\end{lstlisting}

In the GOTO code for this program, there are two significant control flow
operations. The first instruction (line 1) is a GOTO instruction with a
guard\footnote{Even though it's a GOTO instruction, the textual representation
always begins with IF on account of it being a conditional GOTO.}, then the
RETURN instruction on line 6, with label ``2''.

As alluded to in previous subsections, we encode complicated and
nondeterministic executions of the program as small deterministic executions,
and then merge them together later. The storage for such an execution is a
\url{goto_symex_statet::goto_statet} object, which contains a copy of
the L2 variable assignment data, pointer tracking set, and a guard for whether
this path is executed.  Taking our example, at the first GOTO instruction (line
1) we duplicate the current symex state into a \url{goto_statet}, with a
guard evaluating to true when the GOTO's guard is true. This object is then
placed in the stack frame's \url{framet} object, in the
\url{goto_state_map} field, as a piece of program state to be considered
in the future, at the target of the jump (line 4).

Proceding from that GOTO instruction, the guard of the currently executing
piece of code is adjusted to be true when the current instruction is reachable,
i.e. when the guard for the GOTO on line one is false.
We encode the assignment to \url{a} as normal. We then reach the next
GOTO on line 3, which is an unconditional jump to label 2. At this point we
duplicate the current symex state into another \url{goto_statet} object,
and file it for consideration at the jump target (line 5). The guard encoded
in the \url{goto_statet} is the guard for the current execution; we do not
alter it as the jump is unconditional.

ESBMC proceeds by setting the state guard of the current exploration to
false --- this because any instructions following an unconditional jump cannot
be executed. However when it moves onto the next instruction, in the course
of the method \url{goto_symext::merge_gotos} we detect that a previous
jump (from line 1) targetted the current instruction, and thus we should
consider continuing the execution down that path. Given that the current state
guard is false and thus nothing is actually being executed at the current
point of exploration, the state in the \url{goto_statet} from line 1 is
simply copied back into the current exploration state, and the current
exploration guard set to the guard in the \url{goto_statet}. We now
effectively have the program state that we would had if we had actually
jumped the exploration from line 1 to line 4. The assignment in line 4 itself
is executed as normal.

The truly advanced part occurs when we move on to line 5. This is the point
where the two deterministic paths that we could take from line 1 must be
merged into a nondeterministic program. This is signified by the fact that the
current exploration guard is non-false, and \url{goto_symext::merge_gotos}
finds a \url{goto_statet} to merge into the current exploration. To handle
this, the method \url{goto_statet::phi_function} is invoked to merge the
two states, nondeterministically. It begins by taking the difference between
the two guards of the two explorations. It then iterates over all L2 variable
assignments, and finds those that have been changed in either path. For such
assignments, an additional assignment to the variable is made from an
\url{if} irep\footnote{or the if-then-else operation, or a phi-function, or
however you wish to call it}, with the guard difference as the condition, and
the variable from either exploration path as the two operands. The net effect
is that the value of any particular variable in the program changed over the
two explored paths is now selected according to the truth of the condition
on line 1. If it was true, we take the variables assigned in the true path,
if it was false, then the variables assigned in the false path. We finally
take this new L2 variable state, the union of the two pointer tracking sets,
the disjunction of the guards for the paths we could have taken, and assign
this back to the current symex exploration state.

All control flow concepts are built on top of this principal. The unwinding of
loops occurs by ESBMC following all jumps backwards (up to the unwind bound),
with all paths that break out of the loop being filed as states to be merged
once we have exited the loop. Then once we have reached the unwind bound and
exited the loop\footnote{or statically determined that it always exits} these
states are all merged following the procedure as above, and the program
exploration continues.

Special actions are taken during loops, due to this being a bounded model
checker. As expected, we record how many times we have unwound a particular
loop, and once the unwind bound given by the operating user has been reached
we call the \url{goto_symext::loop_bound_exceeded} method. In all
circumstances exploration does not continue any further. However by default
we encode an assertion that the condition of continuing the loop, i.e. the
backwards jump we are currently considering, is false. This causes any path
where additional unwindings are available, but have been ignored due to the loop
bound, to trigger an assertion, letting the user know that their analysis is
incomplete. This can be optionally disabled. If it is, we instead encode an
assumption that the current (artifically bounded) path is not taken. This
preserves the termination condition of the loop, ensuring that only executions
that fufil the termination condition actually leave the loop. If this assumption
is turned off\footnote{The legenday \url{--partial-loops} option} then
ESBMC can consider unsound paths, for example an infinite loop with no
\url{goto} or \url{break} statements will be able to exit and continue
execution because the termination condition is not preserved.

\subsubsection{Function calls, ends, and returns}

There are three function-related GOTO instructions, FUNCTION\_CALL,
FUNCTION\_END, and RETURN. All deal with organizing the exploration of
the program execution into sequential instructions, sometimes involving
some nondeterminism. Consider the simplest case:

\begin{lstlisting}
int main(int argc, char **argv) {
  return foo();
}
\end{lstlisting}

Here, a call to main will be represented as a FUNCTION\_CALL instruction.
The \url{code} field contains a \url{function_call} irep that
records what arguments to pass in, the function target (in a \url{symbol}
irep), and the variable to assign the return value to. Interpretation begins
in \url{goto_symext::symex_main}, where we may identify a function as
being an intrinsic, and pass it off to \url{goto_symext::run_intrinsic}.
All other function calls are passed to
\url{goto_symext::symex_function_call} and then
\url{goto_symext::symex_function_call_symbol}.

As part of setup for a function call, we check that the target function
actually exists. Within ESBMC it is valid to call a declared function that has
no body, otherwise we would have to write stubs for every library ever. If no
body exists to be called, a warning is printed to the user, and the return value
of the call becomes nondeterministic. Additionally we check how many times
the target function has been called to detect recursion; if the number of
active function calls exceeds the unwind bound, the function call is ignored.

Following this, if we are still interpreting the function call, we allocate a
new \url{framet} object to recall this call, and put it on the top of the
call stack. The tracking data for L1 names is updated so that local variables
get renamed to an L1 name local to this function call, rather than a call to
the same function higher on the stack (\url{goto_symext::locality}).
The arguments to the function are iterated over, and assigned to the relevant
local variables in the called function. A small amount of symex time type
coercion also occurs. Finally we store which instruction to return to, what
variable the return value should go to, and update the program counter to the
first instruction in the target function.

In the course of the functions execution we may run into many RETURN
instructions. Each of these are encoded as an assignment to the return value
in the calling function, stored in the \url{framet} object. Following this
we encode an unconditional jump to the end of the function.

Once all paths in the function have been explored we inevitably end up at the
last instruction in the function, possibly with a large number of
\url{goto_statet}'s to be merged in. The final instruction itself is
always a END\_FUNCTION. When this is executed, we reset the program counter to
the calling function and dispose of the \url{framet} on the top level of the
stack.

Most of this is straight forward. It becomes more complicated when calls to
function pointers are involved. In the past, the static pointer analysis
has replaced all function pointer calls with code that takes the set of all
functions that may be pointed at, and encoded a mishmash of conditional GOTOs
and function calls. This results in all the functions the pointer may point at
being called, and the correct outcome of the function call being determined
by nondeterministic merging of states (see the interpretation of GOTOs). More
recently, in a slow attempt to remove the static analysis, I've written code
to dereference function pointers at symex time instead. This takes the set of
pointed-to functions from the symex-time pointer tracking, and prepares function
calls to each of them. One is started, then when that function returns the next
function in the list is called. Once there are no more functions to call,
states are merged in the same way as before. There's no massive benefit from
this, but it will permit more control over function call interpretation in
the future. Relevant fields of the \url{framet} object are
\url{cur_function_ptr_targets}, \url{function_ptr_call_loc},
\url{function_ptr_combine_target}. and
\url{orig_func_ptr_call}.

\subsubsection{Other}

``OTHER'' instructions, as mentioned, are more or less the catch-all instruction
for special cases and ugly hacks. Interpretation of them is redirected to the
\url{goto_symext::symex_other} method. A general description of what
flavours of OTHER instructions there are lies in the section on GOTO
instructions. Dynamic memory operations are described elsewhere. In an ideal
world we would end up replacing all of these special case instructions with
function call to intrinsics.

\subsubsection{Exception handling}

I know practically nothing about the exception handling (and hopefully others
can write this section). However there are THROW\_DECL instructions that I
believe signify the start of a catch block, with THROW\_DECL\_END signifying
the end of a particular type being caught. There's then the THROW instruction
itself, which probably does the obvious.

\subsection{Multithreaded interactions}

The current way that ESBMC operates is to consider an interleaving wherever
access to a piece of shared state occurs. The context switch itself is defined
to occur \textit{after} the operation that causes it. This doesn't affect
reachability at all so long as there is a context switch available at the start
of any thread. This is because the context switch can actually occur at any
point between two accesses to shared state, it's just most convenient to do
it whenever one of those accesses occurs.

To support this, several \url{goto_symext} methods are overridden in the
\url{execution_statet} class. Then, whenever any assignments, asserts,
assumes, and conditional jumps occur, the normal interpretation is still
applied, but is followed by a call to the \url{reachability_treet} object
that controls the current exploration, to see whether any action should be
taken. What action is actually taken depends on the multithreading exploration
method, see another section.

Two additional instructions are available for interpretation by
\url{execution_statet::symex_step} --- ATOMIC\_BEGIN and ATOMIC\_END.
These flag the currently executing thread as being in an atomic block,
and not in an atomic block, respectfully. Other threading operations are
available, but implemented in intrinsics rather than specific instructions.
(I'd also prefer atomic begin / end to be intrinsics too).

The \url{execution_statet} class itself can perform many other operations,
reporting facts about the current program state or making operations like
switching which is the current active thread. However these are all called
from \url{reachability_treet}, so I'll describe them elsewhere.

\subsection{Intrinsic library functions}

Numerous intrinsics exist, mostly related to multithreading operations.
Hopefully in the future more features will move to being intrinsics, thus
reducing special cases and being more obviously ESBMC-specific. All intrinsics
are prefixed with ``\_\_ESBMC\_'', and include the operations below:

\begin{description}
\item[yield] Force a context switch point during exploration
\item[switch\_from] Force a context switch point \textit{and} ensure that
another thread runs at least once before this runs again.
\item[switch\_to] Takes one integer argument. Forces a context switch from the
current thread to the thread identified by the argument.
\item[get\_thread\_id] Return the integer identifier of the current thread.
\item[set\_thread\_data] Takes two arguments. The first is a thread ID num,
the second  is a piece of constant value data to set as thread-specific
data for that thread. Used for sending data to a thread without nondeterminism
getting in the way.
\item[get\_thread\_data] Takes one argument, a thread identifier. Retrieves
a piece of data set with \url{set_thread_data}.
\item[spawn\_thread] Creates a new thread. Takes one argument, the address
of a function to start executing at. This start function must take no arguments
and return nothing. The return of this intrinsic is the ID number of the
created thread.
\item[terminate\_thread] Terminate execution of the current thread, causing it
to never be executed again.
\item[get\_thread\_state] Takes one argument, a thread ID. Returns true boolean
if the thread is still running, false if it has exited.
\end{description}

These are fairly peculiar as far as intrinsics go --- the justification is that
we should be able to build other threading libraries on top of ESBMC that aren't
just pthreads, and so there should be a simple generic interface to controlling
threads. The pthread library implementation on top of this is probably out of
scope for this document, and I'll end up sticking it in the pthread
implementation file.

\section{Dynamic memory}

Currently, if you malloc a piece of dynamic memory what you'll get
back is a pointer to a uniquely numbered variable, probably of type byte array.
Some CBMC code attempts to determine what type variable the caller wants to be
allocated when calling malloc, however this is difficult to do, particularly
when the first call to the result is \url{memset}.

Being a variable means that malloc'd items of data can be treated like any other
piece of data in ESBMC. The complexity occurs when we have to perform assertions
and checks to see whether or not pointers have been freed or point in bounds.
A series of arrays of data are used to model that information. These are
\url{__ESBMC_alloc}, \url{__ESBMC_deallocated},
\url{__ESBMC_is_dynamic} and \url{__ESBMC_alloc_size}.
The index of the array is the pointer object ID number, see the section on
SMT encoding. Here, the boolean arrays \url{alloc} and \url{deallocated}
store whether or not the pointer is valid and has been deallocated,
respectively. Exactly what the difference is between the two is a mystery,
and possibly nonexistant. \url{is_dynamic} is an array of bools indicating
whether or not a particular pointer was allocated dynamically or not.
\url{alloc_size} contains the size of the allocation, if dynamic.

Assertions about pointer validity tend to be written in terms of these pieces
of data. For example, the size of a piece of data is always available, unless
it was allocated dynamically, in which case the \url{__ESBMC_alloc_size}
array is used to retrieve the actual size allocated. Currently no attempt is
made to assert that stack objects that are out of scope are not accessed.

Checking that dynamic memory is all freed before program termination is now
handled by keeping a set of all allocated pieces of memory in
\url{goto_symext}. At the end of the exploration of the program, an
ssertion is encoded that for all objects allocated, if the guard was true
at the time of allocation, then the object has now been freed.

\section{SMT encoding}

Once we have an SSA program to operate upon, we have a series of assignments,
assertions and assumptions. These are then converted into an SMT representation
for checking whether any of the assertions are satisfiable.

At a high level, variables being assigned to are turned into variable names,
and we assert that the right hand side of an assignment is equal to the symbol
name on the left hand side. Variables on the right hand side are turned into
symbols too. To encode the conditions we are checking, we then assert that if
all assumptions are true, it implies that the disjunction of inverted assertion
conditions is true. This causes the formula to be unsatisfiable, unless there
is at least one assertion condition that is false, and all the assumptions are
true (which indicates an assertion failure).

From this higher layer, we then reach the topic of how our SSA programs are
encoded into SMT. We have two logics to use: QF\_AUFLIRA and
QV\_AUFBV\footnote{Although more accurately, the logic we use is whatever Z3
lets us get away with}, referred to as integer mode and bitvector mode.
The primary difference between the two of these is that
integer-mode integers have unbounded range, meaning they overapproximate normal
integers, but have a significant performance boost when solving the formula.
In comparison, bitvector-mode integers are fixed width bit vectors.
The former also represents floating point numbers as reals, wheras the latter
has to approximate floating point numbers with fixed width
arithmetic\footnote{and it's likely all broken, I don't understand it}. However;
the largest difference is that the unbounded range of integer-mode integers
means there can be no accurate byte model of the data being worked on. For
example, if we select the second byte from an unbounded number, what is the
result?

SMT does not support converting between the two representations of integers.
Z3 however has an operation that pupports to do this, but in actuality is an
uninterpreted function. Unfortunately this is used in a number of places
around ESBMC.

Onwards to more complex types. Structs are represented by tuples; the number
of fields is the size of the tuple, and each field is an element in the tuple,
in the order the fields were declared in the struct. Access to struct fields
is then simply selecting or updating a tuple element.

Arrays are a supported type in both logics we use, therefore array operations
can be translated to SMT array operations directly. Note that while C supports
multi dimensional arrays, we translate these to accessing a single dimension
array in the normal manner.

Unions are much more tricky. It appears that the solver backend code keeps track
of what element in a union was most recently assigned to, and when the union
is read in the future, that element is used. However, this is not amenable
to nondeterminism, as an assignment to a union nondeterminstically through an
\url{if} irep cannot be tracked this way. Unions are, thus, broken right
now in the presence of nondeterminism.

Pointers are also sticky. The chosen encoding for these is a two-tuple, where
the first element is a unique integer identifying a particular piece of storage,
i.e. the item pointed at. It is often referred to as the ``pointer object ID''.
The second element is an offset into the pointed at item. Pointers are created
When an \url{address_of} operation is applied to a variable. The SMT
converstion code tracks all variables that have had their address taken, and
when translating an \url{address_of} will return a pointer record with the
(possibly newly allocated) pointer object ID, and a zero offset.

The funkyest part is the casting from a pointer type to an integer type, and
vice versa. This is perfectly legitimate in C languages and is how some pointer
arithmetic occurs. It means however that we have to formulate an address
space in the range of integers, give objects that are pointed at positions
in it, and convert between pointer tuples and integers. We perform the first
part by taking all pointer objects and their size, and defining a symbolic start
and end position in the address space. We then assert that every pointer object
there is does not overlap any other in the address space. The formula for this
is distressingly quadratic, but never mind. We also create an array mapping
pointer object IDs to the base address of that object, which lets us turn
a cast from a pointer object to integer into an array index to find the base
address, then an addition of the pointer offset. Casting from integer to
pointer object is the most distressing --- we must enumerate all pointer
objects and check whether the integer lies in its range.

From a more practial point of view, the base class of SMT conversion is
\url{prop_convt}, which all solvers subclass. We'll only consider Z3 here.
Actual conversion occurs by feeding a \url{symex_target_equationt} object,
which stores all SSA steps, a \url{prop_convt} object which it uses to
encode a series of assertions. Flattening assignments, assumptions and
assertions to just a series of solver assertions happens at this level. The
\url{prop_convt::set_to} method is used to encode an assertion, converting
the expression argument to SMT, and asserting that it evaluates either to
true or false.

Within the solver conversion code, most boolean operations are handled by
solver-agnostic code, but any expression involving a non-bool type must be
handed to the solver specific conversion routines. There, we just have a
gigantic if else block that identifies what operation an irep is, and calls
the appropriate conversion method to turn the operation into a piece of SMT.

\section{Solving and counterexamples}

The act of solving is not massively complex. After the translations described
in the section on SMT encoding have occured, we need only ask the solver
whether or not it is able to make an assignment to variables that satisfies
all assertions. If the result is unsat, then clearly we have little to be
concerned about. However when the result is sat and we have a model of the
formula, with assignments of values to all variables, then we have discovered
a program trace that violates an assertion and we must print a counterexample
to the user.

One irregularity note is that the solver does not calculate assignments to
\textit{all} the variables in the formula being checked. Intead, only the
variables that are actually fundemental for satisfying assertions are
decided upon. Variables in paths that are not part of the failing trace do
not need to have their value decided by the solver, therefore they aren't.
Even better, instructions that are sliced out (see the section on slicing)
will not have been encoded, and thus there will be no assignments for even
more variables. This can lead to confusing and uninterpretable counterexamples.

To actually format a counterexample, high level code calls the function
\url{build_goto_trace}. This takes a list of SSA steps in a
\url{symex_target_equationt} object and strips out certain steps that are not
of interest. These are assignments where the guard of the assignment is false
(i.e., the code block it was in was not executed in is not part of the failing
trace) or internal modelling assignments. The function then inspects each
SSA step, and for each expression that makes up the step (i.e. the rhs,
or guard) it queries the solver to extract from the model the concrete value
of the expression. i.e., what value the solver assigned. These concrete values
are then stored into \url{goto_trace_stept} objects, which are equivalents of
SSA steps but with concrete values. All SSA steps we have not discarded are
converted thus and stored in a \url{goto_tracet} object.

From this point it is a matter of data presentation, the list of
\url{goto_trace_stept} object are iterated over and printed. For assignments,
Location information (of which file/line the instruction came from) is printed
from the instruction, as well as a stack trace, followed by a representation
of the assignment itself. If the solver didn't give a value for a particular
variable assignment however, the extremely helpful message ``assignment
removed'' is printed instead of a variable value. Assumption instructions are
not printed. Assertion instructions are not printed, unless one is found where
the guard is false. When this is found, we print a series of messages about
the assertion that has failed, and the cease printing further trace steps.

One major failing in ESBMC so far is having no useful or canonical
representation in counterexamples of assignments to structs, unions and arrays,
without making either a confusing or entirely inaccurate mess of braces.
Additionally because in the symex phase we flattened all assignments to make the
lhs a variable, an series of assignments to fields in a structs are printed
as assignments of the entire contents of the struct each time (this is a
difficult problem to solve though).

\section{Multithreading exploration}

Several algorithms exist for exploring the multithreaded state space in ESBMC.
All of these currently operate by keeping historical program states (stored
in \url{execution_statet} objects), switching
which thread is being executed at certain points, and backtracking to a past
state and exploring further when a failing trace is not discovered. The most
significant thing about this is that we only ever store a program state and
order a context switch between threads when there is some kind of shared
state that a thread has accessed. If we were to context switch after every
instruction, including accesses to local only state, we would produce a large
set of identical SSA programs. A proof exists that you need only context switch
when some kind of thread synchronisation occurs, such as access to shared state.

Currently the two methods
\url{reachability_treet::analyse_for_cswitch_after_read} and
\url{reachability_treet::analyse_for_cswitch_after_assign} are called
whenever an access or assignment respectively is made to any variable in the
course of any instruction. If the expression read or written turns out to
contain global variables (i.e. shared state), we consider a context switch at
this point. The method \url{reachability_treet::analyse_for_cswitch_base} is
then used to check numerous facts --- are we currently in an atomic block?
Have we seen this state before (state hashing)? If these are untrue, we then
iterate over all threads, attempting to find a thread we have not already
context switched to from this point and that is not rejected by a partial
order reduction.

Once we have identified a thread to switch to, we duplicate the current thread
state and push it onto the top of a list of historical thread states. It is
then asked to switch to the desired thread, and symbolic execution continues
from there.

Later, once we have reached a point where there are no more threads to run,
we return the produced SSA program to the high level routines. If it asks
for another interleaving to be produced, we start backtracking. First, we remove
the last (i.e., the completed run) program state from the list of historical
states. We then attempt to find a context switch that we have not explored
from the previous program state, and explore taking that context switch instead.
If we have explored all context switches from a program state, we backtrack
futher to the next previous program state. This continues until we have explored
all interleavings that can be produced by the program.

This particular search is the \texttt{depth first search} (DFS), where we run to
a deep point in exploration and the progressively backtrack and explore from
there. Also available is a full program exploration\footnote{The
\url{--schedule} option} where we perform the same DFS, but store all SSA steps
for all interleavings in the same \url{symex_target_equationt}. The presence
of the \textit{context switch id} (see the section on variable renaming) in
variables means that every unbroken run of SSA steps has its own identifier,
and its assignments cannot conflict with any other interleavings assignments.
In addition to this there is a round-robin scheduling method, which I'm mostly
unfamiliar with.

\section{The inevitable 'misc'}

There are a large set of things in ESBMC, not all of which have a particularly
important purpose, or are shared between with several. Here we cover such
things.

\subsection{Naming and namespaces}

As defined by the C standard, ESBMC has a concept of naming scopes. Variables
declared in one scope are not necessarily accessible from another. The substance
of this is that any variable declared receives a symbol name that contains
a series of double-colon (::) seperated components that specify its scope. All
symbols start with ``c::'' to seperate program symbols from any internal
ESBMC modelling variables. From there, if the symbol is of internal linkage
(i.e. declared ``static'') it has the name of the C file it originated as
the next scope component. Then follows the global name of the symbol, i.e.
a global variable name or function name. For automatic allocation variables
(i.e. block local variables) a series of numbers are inserted into the
name to identify \textit{which} block in the function the variable is in, so
that variables with identical names declared in different blocks can co-exist.
Finally, the name of the variable is appended. Some examples:

\begin{lstlisting}
c::main
c::beards
c::test::ponies
c::main::1::varname
c::main::1::1::1::varname
\end{lstlisting}

Here, in order, we have a function symbol name, a global variable name,
the name of a global variable declared \texttt{static} in a file called test.c,
a local variable in the main function, and another local variable of the same
name deeply nested in a series of blocks. Note that I don't know how CBMC
decides what numbers to inject into the names of local variables.

Most variables get thrown around ESBMC as \texttt{symbol} ireps, with the
appropriate name as above with additional renaming data. To look up date about
a particular variable, use the \texttt{contextt} object to fetch a
\texttt{symbolt} object for the desired variable. The \texttt{symbolt} object
stores the variables type, lifetime, scope, value (untranslated AST from
parsing) and numerous other things.

\subsection{Dereferencing}

Dereferencing occurs at numerous points in ESBMC, during the initial pointer
analysis to determine what pointer assertions to add, and during symbolic
execution to actually read data through a pointer. To perform a dereference,
we require a \url{dereference} irep with a pointer object as an operand,
an a \url{value_sett} object tracking what pointers may point at. A
\url{dereferencet} object is created and the \url{dereference} method in it
called.

One notable feature of the \url{dereferencet} class is that it takes a
\url{dereference_callbackt} object as a constructor argument, the
\url{dereference_failure} method of which is called when a dereference
failure occurs. This allows the dereferencing code to be notified about
dereference failures and handle them accordingly. The failure itself
is a textual description and a guard for when the the failure will occur,
for example when the pointer object does not match any of the expected
pointers.

In the actual dereference operation, we look up what pointer object is being
dereferenced, and from the pointer tracking value set derive what it might
possibly point at. Then a ``switch case'' is calculated, a large chain of
\url{if} ireps, with guards that test whether the dereferenced pointer object
matches a particular variable that is pointed at, and evaluate to that variable
if this test is true. Thus the resulting expression evaluates to whatever
the pointer points at, encoded symbolically. FIXME: does this make any sense?

The frequent complication that can occur along the way is a type mistmatch.
If we are selecting an integer out of some structure, but the pointer object
can point at a wide range of variables, we will end up with an expression
that attempts to select an integer out of, say, a float, or across a struct
boundry. When this happens we can perform some limited type coercion, but
will fall back to creating a \url{byte_extract} irep if that fails. A
\url{byte_extract} irep represents the operation of extracting a particular
type of data from some incompatible object, at a particular offset. Rather than
attempt to juggle a byte representation of the incompatible object at this
level, we leave it to the conversion to SMT.

\subsection{Slicing}

One notable fact about the programs being tested in ESBMC, is that not all
variables and assignments actually contribute to the assertions being checked,
and are thus a waste of memory and processing time. To work around this problem,
CBMC introduce something known as ``slicing''. Here, once we have a list of
SSA steps, we work through the steps backwords noting down what assertions
occur, and what variables are used in these assertions in a set called
\textit{depends}. We continue working through the steps backwards, and whenever
we find an assignment to a variable that is not in the depends set,
we mark it as being ``ignored'', while variables that \textit{are} used in the
depends set have the variables they are calculated from added to the depends
set.

The net upshot of this is that we progressively calculate what variables
actually contribute to the evaulation of an assertion, and mark all other
variables as being ignored. Then, when we come to conversion to SMT, we ignore
all varaibles that do not contribute. Unfortuantey this option is turned on by
default, and can lead to significantly confusing counterexamples being printed
as ESBMC is quite entitled to discard the majority of an execution trace if
none of it contributes to an assertion.

\subsection{irep}

I could write a large amount of stuff about the internal representation,
how it is structured and what each piece of it \textit{means}. However, that
wouldn't be useful in a manual, so I'll describe the substance of it instead.
At its center, ireps are supposed to describe some form of operation, with
operands, any require annotative information, and a type. This has been further
abused so that the same data structure stores types, program statements, and
in fact the entire AST.

The base type is a \url{irept} object. This stores one primary piece of data,
the ``identifier''. This is a string, and represents what the operation of
the irep is (like \url{add}, \url{dereference}, \url{byte\_extract} and so
forth). \url{irept} also stores a vector of \url{irept}s as numbered operands;
a map of annotative information (mapping strings to more \url{irept}s), and
another map of ``comments''. The point of this, is that \textit{every piece of
data} in an irep is stored as the ``identifier'' string in an \url{irept}
object, and are arranged in operands and annotations in some manner to represent
an operation. Note that the strings themselves get stored into a string pool
to avoid massive memory inflation.

This by itself is fully offensive\footnote{Amounting to a war crime, as far as
I'm concerned} as it seeks to replace normal classes and objects with something
that isn't type safe. There are a few benefits (you can serialise, print, hash
and search data simply as it's all the same type) but the speed and type-safety
penalties are not worth it. In the most insane example, constant numbers are
stored as an \url{irept} with identifier \url{constant}, and the annotative
value \url{value} set to a binary representation of the number it represents.
As in, it's a series of zeros and ones, in ascii.

There are a series of methods that the reader is likely to encounter. First
there is the \url{id} method, which returns the string identifier for the
irep. The \url{operands} method returns the vector of operands, and \url{op0},
\url{op1} and so forth methods accessing specific elements in it. The
\url{find} and \url{add} methods find and set respectively annotative
information by name. As well as these methods, there are additional classes
such as \url{exprt} and \url{typet} that subclass \url{irept}, do not store any
additional data, and essentially act as wrappers around an \url{irept} to
provide methods for easier accessing of the data it stores. One can simply
cast any \url{irept} to an \url{exprt} and treat it as an expression.

The one vastly mis-understood and undocumented fact about \url{irept}s is that
they are reference counted, and \url{const}ness of the object affects its
operation. The only data stored in an \url{irept} is a pointer to an object
storing all the fields described above; the \url{irept} object itself
essentially contains it. Then, whenever a \url{const} copy of an \url{irept}
is taken, the reference count of the storage object is increased. This allows
\url{irept}s to be passed around as objects without duplicating them; it also
means that if we re-use subtrees of an \url{irept} somewhere then the subtree
is not duplicated, it just has an increased reference count.

Clearly sometimes though we actually alter \url{irept}s. When either a
changing operation occurs, or a non-\url{const} \url{irept} is constructed from
another irept, the contents of the \url{irept} is duplicated and stored in
the \url{irept} that is going to be changed. This way, we can have the shared
data benefits described above, but without risking data corruption or
inconsistency. It's actually quite elegant, and a redeeming feature of something
that's absolutely terrible.

\section{Conclusion}

I'm covered in bees, and so can you.

\end{document}
