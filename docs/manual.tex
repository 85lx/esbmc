\documentclass{article}
\usepackage{url}
\begin{document}
\author{Jeremy Morse}
\title{A manual on some internals of ESBMC}
\maketitle
\section{Introduction and caveats}

This ``manual'' is supposed to be an introduction to how the ESBMC model
checkers internals are arranged and operate. It is not supposed to be a
comprehensive piece of documentation on the exact behaviour of particular
functions or facilities; nor will it ever, ever be up to date. Exact
documentation on a particular function or method should be written in
doxygen in headers; this documentation can be built to HTML by executing
\texttt{make doxygen} in the top level directory of ESBMCs source tree.

When referencing portions of code from within this manual, I'll probably
end up referring to class names and methods within them. Source files and
line numbers are liable to change, wheras the code layout of the project
is the least likely to suffer significant churn. The location of such a
class or method should be obvious from the context, or discoverable with
grep.

A huge amount of the code base is derived from the CBMC project. CBMC is
open source (BSD 4-clause, ish), and available over SVN at
\url{http://www.cprover.org/svn/cbmc}. A large number of design decisions
are down to the development of CBMC; changes to ESBMC that cause
significant divergance from CBMCs design should be carefully thought
about, seeing how it's more mature than ESBMC. Likewise, code being
pulled in from CBMC should be examined to see whether it'll actually fit
into what ESBMC is doing nowdays.

All additional gunge, queries, complaints, to \url{jeremy.morse@gmail.com}

\section{Source tree structure}

In an order vaguely related to how ESBMCs execution order occurs, the
following describes the contents of directories in the source tree.

FIXME: This is ugly, make it a table.

\begin{description}
\item[docs] Directory for not-in-code documentation.
\item[papers] Self explanatory.
\item[scripts] Various scripts and auxilary files related to building ESBMC
               and dealing with things that aren't source files. Makefile
               scripts and release/binary manipulating scripts.
\item[esbmc] Top level model checking control code. Process entry point,
             option handling, general direction and invocation of the rest
             of the code base.
\item[langapi] Abstractions for handling input source files. Links a variety
               of global functions up to input-language-appropriate routines.
               Probably not massively necessary and could be ditched.
\item[ansi-c] Parser for ANSI-C software. Contains all code required to lex,
              parse, store as an AST, typecheck, and link, an input file.
\item[ansi-c/cpp] C preprocessor - an import of the Portable C Compilers
                  preprocessor, adapted to do what ESBMC needs.
\item[ansi-c/headers] C langauge headers to override system headers.
\item[ansi-c/library] C language implementations of various code libraries
                      that we seek to model.
\item[cpp] Parser for C++ language. Code for all compilation steps of C++.
\item[cpp/library] Implementation / models of various C++ template libraries.
\item[big-int] Arbitary length integer library. Used internally to avoid any
               kind of problems modelling large machine integers using small
               machine integers.
\item[goto-programs] Routines for general operations on GOTO instructions, as
                     well as all the code for converting a parsed AST into
                     GOTO instructions.
\item[pointer-analysis] Code for interpreting the execution of GOTO instructions
                        and the analysis of their effect upon pointer tracking.
                        Basically a static analysis of pointer assignment and
                        reachability. Also, contains code for resolving pointer
                        indirection in dereferences.
\item[goto-symex] Symbolic execution of GOTO instructions into an SSA program.
\item[solvers] Encoding of SSA program into SMT solver logic, and solving of
               the produced SMT formula.
\item[util] Miscellaneous functions, classes, and whatever to glue everything
            else together.
\item[regression] Regression tests for various different facets of ESBMC.
\end{description}

\section{Top level procedures}
\section{Source file parsing}
\section{GOTO instructions}
\section{Pointer analysis}
\section{Symbolic execution}
\section{SMT conversion}
\section{SMT encoding}
\section{Solving and counterexamples}
\section{The inevitable 'misc'}
\section{Conclusion}
\end{document}
