\documentclass{article}
\usepackage{url}
\begin{document}
\author{Jeremy Morse}
\title{A manual on some internals of ESBMC}
\maketitle
\section{Introduction and caveats}

This ``manual'' is supposed to be an introduction to how the ESBMC model
checkers internals are arranged and operate. It is not supposed to be a
comprehensive piece of documentation on the exact behaviour of particular
functions or facilities; nor will it ever, ever be up to date. Exact
documentation on a particular function or method should be written in
doxygen in headers; this documentation can be built to HTML by executing
\texttt{make doxygen} in the top level directory of ESBMCs source tree.

When referencing portions of code from within this manual, I'll probably
end up referring to class names and methods within them. Source files and
line numbers are liable to change, wheras the code layout of the project
is the least likely to suffer significant churn. The location of such a
class or method should be obvious from the context, or discoverable with
grep.

A huge amount of the code base is derived from the CBMC project. CBMC is
open source (BSD 4-clause, ish), and available over SVN at
\url{http://www.cprover.org/svn/cbmc}. A large number of design decisions
are down to the development of CBMC; changes to ESBMC that cause
significant divergance from CBMCs design should be carefully thought
about, seeing how it's more mature than ESBMC. Likewise, code being
pulled in from CBMC should be examined to see whether it'll actually fit
into what ESBMC is doing nowdays.

All additional gunge, queries, complaints, to \url{jeremy.morse@gmail.com}

\section{Source tree structure}
\section{Top level procedures}
\section{Source file parsing}
\section{GOTO instructions}
\section{Pointer analysis}
\section{Symbolic execution}
\section{SMT conversion}
\section{SMT encoding}
\section{Solving and counterexamples}
\section{The inevitable 'misc'}
\section{Conclusion}
\end{document}
