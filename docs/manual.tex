\documentclass{article}
\usepackage{url}
\usepackage{listings}
\begin{document}
\author{Jeremy Morse}
\title{A manual on some internals of ESBMC}
\maketitle
\section{Introduction and caveats}

This ``manual'' is supposed to be an introduction to how the ESBMC model
checkers internals are arranged and operate. It is not supposed to be a
comprehensive piece of documentation on the exact behaviour of particular
functions or facilities; nor will it ever, ever be up to date. Exact
documentation on a particular function or method should be written in
doxygen in headers; this documentation can be built to HTML by executing
\texttt{make doxygen} in the top level directory of ESBMCs source tree.

When referencing portions of code from within this manual, I'll probably
end up referring to class names and methods within them. Source files and
line numbers are liable to change, wheras the code layout of the project
is the least likely to suffer significant churn. The location of such a
class or method should be obvious from the context, or discoverable with
grep. Numerous references will also be made to the \textit{Internal
representation}, or \textit{irep} of something. This refers to how some piece
of data is structured or stored, see the 'Misc' section for detail.

A huge amount of the code base is derived from the CBMC project. CBMC is
open source (BSD 4-clause, ish), and available over SVN at
\url{http://www.cprover.org/svn/cbmc}. A large number of design decisions
are down to the development of CBMC; changes to ESBMC that cause
significant divergance from CBMCs design should be carefully thought
about, seeing how it's more mature than ESBMC. Likewise, code being
pulled in from CBMC should be examined to see whether it'll actually fit
into what ESBMC is doing nowdays.

All additional gunge, queries, complaints, to \url{jeremy.morse@gmail.com}

\section{Source tree structure}

In an order vaguely related to how ESBMCs execution order occurs, the
following describes the contents of directories in the source tree.

FIXME: This is ugly, make it a table.

\begin{description}
\item[docs] Directory for not-in-code documentation.
\item[papers] Self explanatory.
\item[scripts] Various scripts and auxilary files related to building ESBMC
               and dealing with things that aren't source files. Makefile
               scripts and release/binary manipulating scripts.
\item[esbmc] Top level model checking control code. Process entry point,
             option handling, general direction and invocation of the rest
             of the code base.
\item[langapi] Abstractions for handling input source files. Links a variety
               of global functions up to input-language-appropriate routines.
               Probably not massively necessary and could be ditched.
\item[ansi-c] Parser for ANSI-C software. Contains all code required to lex,
              parse, store as an AST, typecheck, and link, an input file.
\item[ansi-c/cpp] C preprocessor - an import of the Portable C Compilers
                  preprocessor, adapted to do what ESBMC needs.
\item[ansi-c/headers] C langauge headers to override system headers.
\item[ansi-c/library] C language implementations of various code libraries
                      that we seek to model.
\item[cpp] Parser for C++ language. Code for all compilation steps of C++.
\item[cpp/library] Implementation / models of various C++ template libraries.
\item[big-int] Arbitary length integer library. Used internally to avoid any
               kind of problems modelling large machine integers using small
               machine integers.
\item[goto-programs] Routines for general operations on GOTO instructions, as
                     well as all the code for converting a parsed AST into
                     GOTO instructions.
\item[pointer-analysis] Code for interpreting the execution of GOTO instructions
                        and the analysis of their effect upon pointer tracking.
                        Basically a static analysis of pointer assignment and
                        reachability. Also, contains code for resolving pointer
                        indirection in dereferences.
\item[goto-symex] Symbolic execution of GOTO instructions into an SSA program.
\item[solvers] Encoding of SSA program into SMT solver logic, and solving of
               the produced SMT formula.
\item[util] Miscellaneous functions, classes, and whatever to glue everything
            else together.
\item[regression] Regression tests for various different facets of ESBMC.
\end{description}

\section{Command line options}

Insert here, descriptions on command line options.

\section{Top level procedures}

Entry to the process starts (more or less) in the \texttt{doit} method of the
\texttt{cbmc\_parseoptionst} object. Various command line options are checked
for validity, before the \texttt{get\_goto\_program} method invokes the
frontend parsers to compile input source code into an AST. The AST for the
entire environment (all source files and libraries) is stored in a
\texttt{contextt} object, containing a list of symbols and their AST value.

The contents of the \texttt{contextt} object is passed to the
\texttt{goto\_convert} function, which produces a set of
\texttt{goto\_functiont}s corresponding to each function in the source language.
Each function contains little more than a \texttt{goto\_programt}, which
actually contain a list of instructions and some annotations.

With the set of GOTO functions, the \texttt{process\_goto\_program} method
applies the string abstraction transformation, the pointer analysis,
installs various pointer validity checks, and anything else that transforms
the source program into different instructions (such as LTL property monitors
or data race checking).

With these fixed-up goto functions, a \texttt{bmct} object is created and the
\texttt{run} method invoked on the functions. These functions are fed into a
\texttt{reachability\_treet} object, the primary interface to symbolic
execution. Within the \texttt{bmct::run} method, the symbolic execution engine
is asked to run through instructions creating an SSA program; potentially
several times if there are multiple threads involved. A result itself is a
\texttt{goto\_symext::symex\_resultt} containing the SSA program container and
a count of how many assertions remain to be verified in the program.

The SSA program is then optionally sliced; see the 'Misc' section for details.

A solver object is then created, a subclass of \texttt{bmct::solver\_base}
abstraction which solver to use. The SSA program is then fed to the solver,
which encodes it to SMT or whatever appropritate encoding it uses. It's then
asked to solve the equation, returning:
\begin{description}
\item[UNSATISFIABLE] The formula isn't satisfiable.
\item[SATISFIABLE] The formula is satisfiable.
\item[SMTLIB] A special case for printing the formula to a SMTLIB file.
\item[ERROR] Some error occured during solving.
\end{description}

Finally, if the formula is satisfiable, an error trace is created and printed.
Further details in the 'Misc' section.

\section{Source file parsing}

Code parsing is one of the untouched pastures of CBMC code, mostly. The ANSI-C
frontend is almost entirely like the original, while the C++ frontend has been
significantly developed by Manaus. The author is really familar with neither.
I'll talk about the ANSI-C frontend, then how the C++ frontend relates to it.

There are some significant conceptual steps involved. Firstly consider
the input and the output. Coming in is a C source file that must be preprocessed
and parsed - two fairly straightforward (although not easy) tasks. The top
levels of ESBMC then receive an AST representing the types and code structure
of the source file, which is more complex. The irep / structure of this data
is entirely undocumented and closely coupled between the parsers and the
\texttt{goto-programs} dir that converts it to GOTO instructions. However the
contents of this AST tends to be fairly high level language constructs,
for example \texttt{for} and \texttt{switch} statements. Refer to the method
\texttt{goto\_convertt:convert} for an idea of what kind of constructs these
are. The majority of the source file parsing code deals with converting between
the parse tree and the AST.

The preprocessing stage is contained in the \texttt{c\_preprocess} function.
In CBMC this used to offload preprocessing to the host preprocessor, however
our requirements have become more complicated since then. We now do
preprocessing using the preprocessor from the Portable C Compiler project
(which is BSD licensed). Unfortunately it wasn't designed with memory management
in mind, so ESBMC picks an output file, forks, calls the preprocessor to pump
outupt to the selected file, then exits the child process.

The complicated requirements from the preprocessor is that we generally want
code under test to have access to all headers on the host system, however we
also want to shoehorn our own types and functions in there --- for example the
glibc headers for assert have some obnoxious defines involved. Additionally
given how much lee-way standards give to libraries to define the format of
opaque data structures, we may need to define our own data structures to
avoid having to special case code for different operating systems. For example,
\texttt{pthread\_t}'s, \texttt{pthread\_mutex\_lock}'s and so forth differ
between operating systems, and we currently rely on on using Linux' pthread.h
header. The current fix for this is to intercept \texttt{\#include} statements
and read in an ESBMC specific header file from \texttt{ansi-c/headers} rather
than the system headers.

We parse C in the normal way; a flex tokenizer is defined, and a yacc grammar
defined which translate the input C into a parse tree. This needs no special
description. A class (\texttt{ansi\_c\_convertt}) takes the parse tree and
makes a simple translation to the format of the AST. The bulk of the work then
lies in the \textit{typechecking} phase. Here, the nasty parts of C that are
context-dependant\footnote{i.e., all of them} are fixed up. Factors such as
integer promotion, operation signedness, and actual correctness are
considered, and various casts or extensions are inserted. The output is stored
as a set of symbols with associated AST values in a \texttt{contextt} object.

After parsing and compilation is linking. In the past CBMC has just
tacked all libraries available onto the end of a source file being compiled,
and that's all. Nowdays the libraries are pre-compiled into a binary
representation of GOTO instructions, and linked in after typechecking of C
code, by copying in any symbol referred to from the compiled source files
that are in the compiled library files.

Finally there's the initialization of C global and static lifetime variables.
Seeing how the GOTO language is only made up of assignments, ish, their
initialization must be made by assignments too. So, an initial 'main' GOTO
function is synthesized from the \texttt{c\_main} and
\texttt{static\_lifetime\_init} functions. For each global variable an
assignment is emitted assigning the initial value to the global variable.

\section{GOTO programs}

This section covers both the GOTO program record itself, and the GOTO
instructions that make it up. Before launching into a description of these
records, it is important to understand that CBMC synthesizes the
\texttt{goto\_programt} class from the \texttt{goto\_program\_templatet}
template. This can lead to the most confusing and obscure error messages
if you do not realise that you're manipulating a template. From a design
point of view, the reason for this appears to be so that the types of
the GOTO program body could be parameterised; a decision that is almost
entirely without merit.

The \texttt{goto\_programt} class is more or less just a container for a
list of GOTO instructions (of class \texttt{goto\_programt::instructiont}).
It stores \textbf{no} additional information. Instead, all of its methods
perform operations on the contained instructions. Most of these relate
to the creation, insertion, and deletion of instructions, recalculating
their contents to be consistent after such a modification, and a few
special cases such as determining the successor instructions from a
particular instruction,

The actual GOTO instruction class itself's primary piece of data comes in
two flavours -- the \texttt{code} member or the \texttt{guard} member.
These store the internal representation of what the body of the instruction
\textit{is}. Exactly what the instruction means depends on the \texttt{type}
field, described thus:

\begin{description}
\item[GOTO] Jump from the current instruction to the instruction in the
\texttt{targets} field. If \texttt{guard} is not true, then the jump is
conditional, depending on the evaluation of \texttt{guard}. If \texttt{guard}
is true the jump to the target occurs; if not, execution continues to the next
instruction.
\item[ASSUME] Encode an assumption, stored in the \texttt{guard} field, to the
solver.
\item[ASSERT] Encode an assertion, stored in the \texttt{guard} field, to the
solver.
\item[OTHER] Catch-all instruction for storing special cases, enumerated below.
Identified by what kind of irep is stored in the \texttt{code} field.
\begin{description}
\item[cpp\_delete] Also \texttt{cpp\_delete[]}. Represents a deallocation of
some memory allocated by C++'s \texttt{new} or \texttt{new[]} operators.
\item[printf]\footnote{Yes, really} Represents a printf operation, for later
printing in a counterexample.
\item[decl] Represent declaration of a variable. Normally the declaration of
a variable is uninteresting as we only care about when it is initialized.
However in a loop where a variable is declared inside the loop block, it
transitions from being initialized to uninitialized when the loop iteration
finishes. Hence the importance of knowing where it is declared.
\item[nondet] Represent a nondeterministic value, from a \texttt{nondet\_*}
function call.
\item[asm] Inline assembly statement. Mercifully ignored.
\item[typeid] Fetch a C++ type ID record, I belive.
\end{description}
\item[SKIP] An ignored instruction.
\item[LOCATION] Previously caused a ``location'' SSA step to be recorded for
future tracking of the code path of the counterexample. Now redundant.
\item[END\_FUNCTION] End instruction of a function. Not the same as a return,
which can occur anywhere, but actually the final instruction in the list of
instructions.
\item[ATOMIC\_BEGIN] Self explanatory.
\item[ATOMIC\_END] Self explanatory.
\item[RETURN] Record a return statement, identifying the expression to return.
Stored in a ``return'' irep in the \texttt{code} field.
\item[ASSIGN] Self explanatory. An ``assign'' irep is stored in the
\texttt{code} field, identifying the left and right hand sides.
\item[DECL] Unused. Probably used to be, or was intended to be, the decl
irep from the OTHER instruction.
\item[DEAD] Unused. Comments say ``marks the end-of-live of a local variable''.
\item[FUNCTION\_CALL] A function call record; stores a function call irep in
the \texttt{code} field, which in turn records the left hand side of the call,
the arguments, and the target.
\item[THROW] Throw record; not familiar with this, but it'll result in some
kind of an assignment to a record of what's been thrown, and a jump to somewhere
else.
\item[THROW\_DECL] Record the start of a catch block for a particular type
of variable.
\item[THROW\_DECL] Record the end of a catch block for a particular type
of variable.
\end{description}

All behaviours of GOTO programs are described by lists of these instructions.
Additional annotations are stored with each instruction, for example the
\texttt{function} and \texttt{location} fields identify where in the source
files the instruction came from. The \texttt{targets} list contains a list
of where the instruction can jump to (which should only ever contain zero or
one target instructions).

\texttt{loop\_number} identifies a unique loop number for backwards
GOTO instructions. \texttt{target\_number} is a numeric ID that labels
the instruction within a function. This don't actually do anything, but
is printed in the textual representation of GOTO instructions to indicate
the targets of GOTOs.  There's also a set of local variable names, and
a globally unique instruction ID in \texttt{location\_number}.

That's the substance of instructions; more information on the interpretation
of them lies in the symbolic execution section.

\section{Pointer analysis}

The essence of the pointer analysis is a tracking of what pointer variables
exist in the GOTO code, and what they might point at. This occurs more than once
during each run of ESBMC. A static analysis of the instructions first attempts
to establish a set of all (lexical) variables that a particular (lexical)
variable in a function may point at. Then during symbolic execution, a similar
tracking maintains a set of (``runtime'') variables that an actual pointer
\textit{does} point at.

The static analysis is initiated from the GOTO program processing code in the
\texttt{cbmc\_parseoptionst} object. The high level analysis logic actually
lies in the \texttt{goto-programs} directory with the
\texttt{static\_analysist} and \texttt{abstract\_domain\_baset}
classes. Code in these classes call abstract methods to perform transformations 
between states as appropriate, over all GOTO instructions, to find a fixedpoint
where all values of the abstract domain have been discovered for all states.

The \texttt{value\_set\_analysist} and \texttt{value\_set\_domaint} classes
subclass the above two classes respectively to provide concrete
methods\footnote{I'm probably using all the wrong terminology here} for
tracking states of what pointer variables might point at. Most of the logic
itself lies in the latter class, storing both the actually tracking data
and forwarding transformation method calls to the appropriate objects.

A \texttt{value\_set\_domaint} contains only a \texttt{value\_sett} object.
That itself contains the pointer tracking map, which is, unsuprisingly,
string based. The core type is the \texttt{value\_sett::valuest} map,
where a string identifying a variable name maps to a
\texttt{value\_sett::entryt}, which stores a set of variables that may be
pointed at and the offset into them.

The string key of each of these entries is important -- When interpreting 
an assignment of a pointer value to a variable, we take the original variable
name being worked on and then interpret the left hand side, appending strings
to indicate /what/ part of the variable is being assigned to. To illustrate,
consider an assignment to the \texttt{bees} field of the following struct:
\begin{lstlisting}
struct face {
  void *bees[4];
};

int main() {
  struct face knees;
  knees.bees[0] = NULL;
  return 0;
}
\end{lstlisting}

Here, the fully qualified name of the variable we are assigning to is
\texttt{main::main::knees}, which becomes the starting point for the string
key in the value tracking map. We then interpret the left hand side of the
assignment, observe that we access the \texttt{bees} field, and so append the
text \texttt{.bees} to the key we are calculating. The next part of the left
hand side is the access to an element of the \texttt{bees} array, so we
append the text \texttt{[]} to the key we calculate. The final key is then
\texttt{main::main::knees.bees[]}. Observe that this approach allows every
variable in the program to have a unique key in the tracking map, except for
elements in an array --- we instead track what \textit{all} elements of the
array may point at, thus forming an overapproximation. The reasons for this
should be obvious.

The \texttt{value\_sett::entryt} class is responsible for tracking the target
variables that a pointer may point at. It stores a map between certain variable
names and \texttt{value\_sett::objectt}s. The presence of a variable name key
in the map indicates that the variable may be pointed at. The
\texttt{value\_sett::objectt} object records whether the offset into the
variable that is pointed at is nondeterministic or constant; and if fixed,
then what the offset is. (NB: the actual implementation of this stores
\texttt{symbol} ireps as the map keys. To optimise this, it uses a (global)
pooling technique to assign each irep an ID number; see the
\texttt{value\_sett::object\_numbering} object. The ID number is then used
as the key into the \texttt{value\_sett::entryt} map).

The \texttt{value\_sett} class also provides operations required in the course
of the static analysis, most importantly the ability to interpret an instruction
to record and update the tracking described above. The class can also merge
value set records into each other. This follows the obvious merging procedure;
however when the two tracking sets being merged have a pointer variable
that points at different offsets into the same data object, the merged
tracking set records a nondeterministic offset into that data object. This
forms an overapproximation of the offset into an object that a pointer points
at.

No attempt is made to track what I'll term \textit{funky} pointer assignments.
For example, if code deconstructs a pointer variable into bytes, then
reconstructs these bytes into a pointer value, we are unable to track
what the resulting pointer value points at. How to address this in the future
is an open question. The byte array memory models of other tools neatly
side-step this issue.

The static analysis process eventually reaches a fixedpoint state where we
have established all possible variables that may be pointed at byte pointer
variables. The contents of this analysis is then handed to an object of
class \texttt{goto\_program\_dereference}. This proceeds to enumerate all
GOTO instructions and attempts to perform all dereferences in the instruction.
Pointer safety assertions are then generated (see the section on dereferencing)
and inserted as ASSERT instructions prior to the dereferencing instruction.

The pointer analysis executed during symbolic execution uses the same records
and functions as the static analysis. While the static analysis attempts to find
all the variables a pointer might point at across all code paths, the symex
tracking instead tracks the set of all variables a pointer may point at in the
course of the current code path. The variables it tracks are also ``L1 renamed''
(see the section on Symbolic Execution).

It is speculated that the static analysis can be removed, and assertions
encoded on-the-fly when dereferences occur during symbolic execution. While
this might be a valid optimisation, the TACAS13 performance figures indicate
that execution time is dominated by symex, rather than the pointer
analysis\footnote{660 seconds ``GOTO processing'' compared to 20,000 seconds
``BMC time''}.

\section{Symbolic execution}

This portion of ESBMC is likely the most important in terms of theory,
complexity, and performance. The overall task is to take an input set of
GOTO functions, and interpret their instructions through an execution path
that is bounded according to the rules of BMC. Along the way, a \textit{Single
Static Assignment} (SSA) program is created recording the operations that
occured during the trace, for later checking. We must also make decisions
regarding the exploration of thread interleavings if multiple threads are
involved. To cover this section, we'll first look at the substance of the
SSA program, the concept of variable ``renaming'', the classes that make up
the symex process, how specific instructions are interpreted in single
threaded code including certain special case library functions,
and finally how all this relates to multi-threaded code.

\subsection{SSA programs}

The output from a single symex run through a program is contained in a
\texttt{symex\_target\_equationt} object. This primarily stores a list of
\texttt{symex\_target\_equationt::SSA\_stept} objects, each of which represents
a single operation in program, sometimes referred to as a \textit{trace}.
The idea of SSA \textit{variables} will be discussed in the ``Variable
Renaming'' section. A SSA step comes in four flavours:

\begin{description}
\item[ASSIGNMENT] An assignment to a variable, with a symbol on the left hand
side, and an expression irep on the right hand side.
\item[ASSERT] Represent an assertion that an expression evaluates to true.
\item[ASSUME] Represent an assumption that an expression evaluates to true.
\item[OUTPUT] Record an output from the trace. This is essentially a wrapper
around a printf operation, that causes variables to be printed in the
course of printing a counterexample. I recommend not asking.
\end{description}

Numerous expression fields in a step object record the details of the above
operations, and are generally uninteresting for this discussion. The only
other data of note is that each step stores a \texttt{source} object, recording
where in the GOTO program the SSA step was generated.

\subsection{Variable renaming}

FIXME: this section is likely worded in a cack handed manner, requires
verification.

In the course of this manual there are numerous references to ``variables'',
without futher elaboration. The different aspects of variables that might be
referred to are below:

\begin{itemize}
\item The lexical variable, i.e. the variable name itself in a particular
context, not specific to a particular value or function activation.
\item The storage of a variable. The portion of memory reserved for storing
the value of a variable, either a global variable or memory allocated on the
stack if an automatically allocated variable in a function block.
\item The contents of a variable. Simply the value assigned to the variable
at a particular point in a program.
\end{itemize}

We name these different aspects \textit{level 0}, \textit{level 1}, and
\textit{level 2} respectively, often shortened to L0, L1, or L2. The reason
for this naming will become obvious shortly. An example is in order to fully
understand this. Consider the following function, and how it relates to the
pointer analysis:

\lstset{numbers=left}
\begin{lstlisting}
int anint;

void somefunc(int **beans) {
  int *bears = *beans;
  bears = &anint;
}

int main1(void) {
  int wololololo = 0;
  somefunc(&wololololo);
}

int main2(void) {
  int ponies = 0;
  somefunc(&ponies);
}
\end{lstlisting}

Here, if we consider the variable \texttt{beans} as an L0 variable, what it
points at is the set of all pointer targets it could potentially point at over
\textit{all} execution traces that are available in the program, including
all pointers that may be fed into \texttt{somefunc} at any point in the
program. This set includes the \texttt{wololololo} and \texttt{ponies}
variables. However if we consider \texttt{beans} as an L1 variable, what it
points at is whatever it may point at across the lifetime of its storage. Thus,
this depends on the path taken through the program -- the pointer may point
at \texttt{wololololo} or \texttt{anint} if \texttt{somefunc} is called from
\texttt{main1}, or alternately it may point at \texttt{ponies} or \texttt{anint}
if called from \texttt{main2}. However it may never potentially point at all
three. Finally, L2 variables are the actual value of the variable at a
particular point in the program, in our example the L2 \texttt{beans} pointer
may only ever point at a single value, which varies with both the path through
the program and the actual instruction location.

(These differences are important during the pointer analysis -- the static
analysis tracks pointers as L0 variables, the symex pointer tracking considers
L1 variables).

The reason for describing these variable kinds in terms of ``level''s, is
that we can consider a higher ``level'' of variable as representing a set of
lower level variables. Consider: if we call the \texttt{somefunc} function
in the above example three times, then the L0 \texttt{beans} variable exists
only once (as the function only exists once, so the lexical variable exists
once). However, three L1 instances of the L1 \texttt{beans} variable exist,
because the function was called three times and storage allocated for the
variable three times. In the same way, repeated assignment to a variable
in a function causes only one L1 name to exist (as only one piece of storage
is required), however multiple L2 variables exist as the variable has
multiple values during the execution of the function.

This then leads to the concept of ``renaming'', which is where we take a high
level variable and reference a lower level variable, appropriate to the context.
The two forms this can take are the transitions L0 to L1, and L1 to L2. In the
former case, we are taking a lexical variable and referring to the actual
storage of that variable. If the variable is global or of a static lifetime,
no additional data is added to the record, as the variable is always the
same piece of storage across the whole program execution. If the variable is
allocated local to a function call, then the L0 variable is annotated with
a unique \textit{activation record} number that identifies which invocation
of the function we are dealing with, and a thread number. The actual values
of these numbers are derived from the context of the renaming -- a thread number
is always available, and the activation record number of the function is always
available (as it's illegal to refer to a local variable outside of its
scope\footnote{Indirect pointer references are special, see later}.


When renaming an L1 variable to an L2 variable (effectively finding the value
currently stored by that L1 variable) we annotate the L1 variable with
a context switch ID number (see later sections) and an \textit{SSA assignment
number}. The assignment number is a monotonically increasing counter giving each
assignment to an L1 variable a unique identifier. So, if we assign to a
variable four times in a function, four L2 variables are created with
assignment numbers from one to four. This preserves the SSA constraint that we
only ever assign to a variable once. When renaming an L1 variable to L2 for the
purpose of using its value rather than making an assignment, we take the
assignment number to be the greatest assignment number for that L1 variable,
thus giving us the most recently assigned value of that L1 variable.

The final complexity to this situation, is that there is never a particular
object in ESBMC that represents any of these variables. Instead, variables
are identified by \textit{name}, stored in a symbol irep. Unsuprisingly, this
name is a string\footnote{Changes in the irep2 branch}. Multiple symbol irep
objects can contain the name of the same variable; creating a new variable
at any leve is as simple as creating a new name. An interesting side-effect
of this is that, following the procedure for renaming an L1 variable to L2,
if the L1 variable has never been assigned to then the greatest assignment
number is zero, and so references to unassigned variables read from assignment
number zero. (And because when we reach the solver level there is nothing
constraining the zeroth assignment of that variable, it's a free variable).
The format of this string is as follows:

\begin{quote}
full\_variable\_name@actv\_record!thread\_no\&cswitch\_no\#assign\_no
\end{quote}

Where \texttt{actv\_record}, \texttt{thread\_no}, \texttt{cswitch\_no}, and
\texttt{assign\_no} are replaced with the appropriate numbers. L0 names only
feature the variable name at the start; L1 names follow the above format
up until the ampersand; and L2 names use the full format. These names will
frequently crop up throughout almost all of ESBMC.

\subsection{Class overview}

The top level class for symex is \texttt{reachability\_treet}. This contains,
at some level, all the state involved in the interpretation of GOTO
instructions. It also has the high level entry-to-symex methods like
\texttt{reachability\_treet::get\_next\_formula}. Almost all the actual
logic in the class is related to the exploration of multithreaded interleavings,
everything symex related exists in lower level classes. The main piece of
data stored is a set of \texttt{execution\_statet}'s, each of which stores
the full state of the program at a particular point in time.

The next two classes of interest are the \texttt{goto\_symext} and its
subclass, \texttt{execution\_statet}. The former contains all the logic
for GOTO interpretation and anything else only related to single threaded
execution, plus stores a few parameters for symex exploration. The latter
overrides a number of methods and injects logic for discovering when
multithreading operations must occur. The logic in this class only relates
to operations on the program state rather than interleaving exploration.
It stores a set of thread state objects, of class
\texttt{goto\_symex\_statet}, while some specialised information is broken out
into \texttt{execution\_statet} for easy accessibility (such as what threads
are still running, atomic blocks, startup parameters). Global program state
is also stored here as a \texttt{value\_sett} object to track pointer value
sets, and a \texttt{renaming::level2t} storing data on the latest assignments
to variables.

Within the \texttt{goto\_symex\_statet} class are all thread specific pieces
of data --- things like the program counter (in
\texttt{symex\_targett::sourcet}) or the call stack (in
\texttt{goto\_symex\_statet::framet}). Significantly more state is stored
tracking the nondeterministic exploration of the program.
\texttt{goto\_symex\_statet::goto\_statet} objects contain the whole program
state resulting from a short deterministic path, which are then merged together
where control paths merge\footnote{i.e., SSA phi functions} to form the
nondeterministic program trace. Various objects of \texttt{guardt} class
record the guard of a particular path being taken. Also stored is the count
of how many times a loop has been unwound in the current context, for the
purpose of loop bounding.

A \texttt{goto\_symex\_statet::framet} stores information related to a
particular stack frame / function activation. Mundane facts like the function
name, call site location, return value variable and the names of local variables
live here. More exciting items such as the set of executions to be merged in
the future, and the actvation record number for renaming, live here too.
There are also some hacks related to function pointer interpretation.

\subsection{Instruction interpretation}

The execution of instructions begins in the constructor for
\texttt{execution\_statet} where the first thread is created and its program
counter set to the entry of the \texttt{main} function. From that point
onwards, the \texttt{goto\_symext::symex\_step} method is repeatedly called
to take the next instruction, interpret it, and move forwards. Some
instructions are handled in \texttt{execution\_statet::symex\_step},
which overrides its superclass and interprets multithreading specific
instructions. What follows is a narative of the (not quite) high level
operations performed during the interpretation of these instructions. Assume
that pointer dereferences are already handled as described in the section
on dereferencing --- actions specific to dereferencing will be mentioned
explicitly.

\subsubsection{Assign}

The substance of an assignment is a left hand side and right hand side. The
outcome we need is an SSA assignment, where we calculate a value with a type
n the right hand side, and then create a new left hand side variable that this
right hand side can be bound to. Clearly the assignment of a constant rhs to
a variable on the lhs is the simplest example of this. The operation becomes
more complex when the left hand side is an array index, a struct field,
a cast, or a nondeterminstic symbol due to pointer dereferencing. These
assignments must be rewritten into a simple variable assignment.

Performing this rewriting for an array index, we take the array variable
on the left hand side, and create a WITH operation that updates the desired
element with the right hand side. The assignment is rewritten to become an
assignment of this new value to the array variable. A similar operation
occurs for struct and union member assignments. For a nondeterministic
assignment we encode multiple deterministic assignments, one for each left
hand side, guarded by the appropriate guard. For assignment to a
\texttt{byte\_extract} irep, we replace the right hand side with a
\texttt{byte\_update} irep of the base data object on the left hand side.
The logic for all these operations begins in the
\texttt{goto\_symext::symex\_assign\_rec} method and methods called from there.

Once we have an assignment in this form it is suitable for being an SSA
assignment. However all the variables are L0 variables, and need to be
renamed to refer to the appropriate piece of data or value. In the course
of the \texttt{goto\_symext::symex\_assign\_symbol} method we call
\texttt{goto\_symex\_statet::rename} which identifies each variable on the
right hand side, renames to L1 (storage variables), and then to L2 (actual
values). The identifying numbers for these operations are pulled from the
top level \texttt{goto\_symex\_statet::framet} object of the most recently
called function, and the global L2 state tracking. One notable exception here
is that any operand to an \texttt{address\_of} operation will only be renamed
to L1 --- we do not take the address of a particular variable value, but
instead the storage of the variable.

Once renamed, execution passes to \texttt{goto\_symex\_statet::assignment},
which performs accounting for the assignment on the L2 level. Specifically,
it bumps the SSA assignment number for the variable assigned to, so that future
uses of it will have the updated / recently assigned value. Additionally
we inspect the contents of the right hand side, and if it is a
sufficiently\textsuperscript{\texttrademark} constant value it is cached
in the program L2 state tracking, for constant propagation. Future rename
operations on that variable will have the variable replaced with its constant
value, rather than the variable itself. This method also passes the
assignment to be interpreted by the pointer analysis, which updates what the
left hand side may possibly point at.

Finally the \texttt{symex\_target\_equationt::assignment} method is called
to encode an SSA assignment, the program counter is incremented to point at the
next instruction, and we are done.

\subsubsection{Assert and assume}

ASSERT and ASSUME instructions are no-where near complex. Their substance
is the \texttt{guard} field of the GOTO instruction, which records the condition
that we are asserting or assuming. This expression is renamed as above to refer
to the appropriate L2 or L1 variables, and is passed to the expression
simplifier in case we can determine the truth of the operation statically.
The time and particularly memory benefits of determining this early are
significant.

The relevant methods are \texttt{goto\_symext::claim} for ASSERT and
\texttt{goto\_symext::assume} for ASSUME. These perform the above operations,
and then call \texttt{symex\_target\_equationt::assertion} and
\texttt{symex\_target\_equationt::assumption} respectively.

\subsubsection{Goto}

GOTO is likely the most complex of all the GOTO instructions, due to the sheer
amount of juggling of state that results. The substance of these is again
the \texttt{guard} field of the instruction, and a target instruction where the
GOTO is jumping to. There are two other factors that significant decisions
are made on --- firstly whether the GOTO is \textit{backwards} to an earlier
instruction in the program, and whether the guard of the instruction is (when
simplified) a constant true or false, or nondeterministic.

Consider the naive execution of some GOTO instructions. We start by executing
the first instruction in the program, moving on to the next, and repeating until
we reach the last instruction in the program. We can also handle loops by,
whenever we execute a GOTO that jumps backwards, following that jump and
executing from an earlier instruction in the program. This basic approach to
interpretation is what is employed by ESBMC, with all control flows more
complicated than the above encoded by making portions of the execution trace
nondeterministically executed.

The best way to describe this is through an example: below is a sample piece
of C, and an approximation of the corresponding GOTO code that it translates to.

\begin{lstlisting}
if (somebool) {
  a = b;
} else {
  a = c;
}
return a;
\end{lstlisting}

\begin{lstlisting}
   IF somebool GOTO 1
   ASSIGN a = b
   GOTO 2
1: ASSIGN a = c
2: RETURN a
\end{lstlisting}

In the GOTO code for this program, there are two significant control flow
operations. The first instruction (line 1) is a GOTO instruction with a
guard\footnote{Even though it's a GOTO instruction, the textual representation
always begins with IF on account of it being a conditional GOTO.}, then the
RETURN instruction on line 6, with label ``2''.

As alluded to in previous subsections, we encode complicated and
nondeterministic executions of the program as small deterministic executions,
and then merge them together later. The storage for such an execution is a
\texttt{goto\_symex\_statet::goto\_statet} object, which contains a copy of
the L2 variable assignment data, pointer tracking set, and a guard for whether
this path is executed.  Taking our example, at the first GOTO instruction (line
1) we duplicate the current symex state into a \texttt{goto\_statet}, with a
guard evaluating to true when the GOTO's guard is true. This object is then
placed in the stack frame's \texttt{framet} object, in the
\texttt{goto\_state\_map} field, as a piece of program state to be considered
in the future, at the target of the jump (line 4).

Proceding from that GOTO instruction, the guard of the currently executing
piece of code is adjusted to be true when the current instruction is reachable,
i.e. when the guard for the GOTO on line one is false.
We encode the assignment to \texttt{a} as normal. We then reach the next
GOTO on line 3, which is an unconditional jump to label 2. At this point we
duplicate the current symex state into another \texttt{goto\_statet} object,
and file it for consideration at the jump target (line 5). The guard encoded
in the \texttt{goto\_statet} is the guard for the current execution; we do not
alter it as the jump is unconditional.

ESBMC proceeds by setting the state guard of the current exploration to
false --- this because any instructions following an unconditional jump cannot
be executed. However when it moves onto the next instruction, in the course
of the method \texttt{goto\_symext::merge\_gotos} we detect that a previous
jump (from line 1) targetted the current instruction, and thus we should
consider continuing the execution down that path. Given that the current state
guard is false and thus nothing is actually being executed at the current
point of exploration, the state in the \texttt{goto\_statet} from line 1 is
simply copied back into the current exploration state, and the current
exploration guard set to the guard in the \texttt{goto\_statet}. We now
effectively have the program state that we would had if we had actually
jumped the exploration from line 1 to line 4. The assignment in line 4 itself
is executed as normal.

The truly advanced part occurs when we move on to line 5. This is the point
where the two deterministic paths that we could take from line 1 must be
merged into a nondeterministic program. This is signified by the fact that the
current exploration guard is non-false, and \texttt{goto\_symext::merge\_gotos}
finds a \texttt{goto\_statet} to merge into the current exploration. To handle
this, the method \texttt{goto\_statet::phi\_function} is invoked to merge the
two states, nondeterministically. It begins by taking the difference between
the two guards of the two explorations. It then iterates over all L2 variable
assignments, and finds those that have been changed in either path. For such
assignments, an additional assignment to the variable is made from an
\texttt{if} irep\footnote{or the if-then-else operation, or a phi-function, or
however you wish to call it}, with the guard difference as the condition, and
the variable from either exploration path as the two operands. The net effect
is that the value of any particular variable in the program changed over the
two explored paths is now selected according to the truth of the condition
on line 1. If it was true, we take the variables assigned in the true path,
if it was false, then the variables assigned in the false path. We finally
take this new L2 variable state, the union of the two pointer tracking sets,
the disjunction of the guards for the paths we could have taken, and assign
this back to the current symex exploration state.

All control flow concepts are built on top of this principal. The unwinding of
loops occurs by ESBMC following all jumps backwards (up to the unwind bound),
with all paths that break out of the loop being filed as states to be merged
once we have exited the loop. Then once we have reached the unwind bound and
exited the loop\footnote{or statically determined that it always exits} these
states are all merged following the procedure as above, and the program
exploration continues.

Special actions are taken during loops, due to this being a bounded model
checker. As expected, we record how many times we have unwound a particular
loop, and once the unwind bound given by the operating user has been reached
we call the \texttt{goto\_symext::loop\_bound\_exceeded} method. In all
circumstances exploration does not continue any further. However by default
we encode an assertion that the condition of continuing the loop, i.e. the
backwards jump we are currently considering, is false. This causes any path
where additional unwindings are available, but have been ignored due to the loop
bound, to trigger an assertion, letting the user know that their analysis is
incomplete. This can be optionally disabled. If it is, we instead encode an
assumption that the current (artifically bounded) path is not taken. This
preserves the termination condition of the loop, ensuring that only executions
that fufil the termination condition actually leave the loop. If this assumption
is turned off\footnote{The legenday \texttt{--partial-loops} option} then
ESBMC can consider unsound paths, for example an infinite loop with no
\texttt{goto} or \texttt{break} statements will be able to exit and continue
execution because the termination condition is not preserved.

\subsubsection{Function calls, ends, and returns}

There are three function-related GOTO instructions, FUNCTION\_CALL,
FUNCTION\_END, and RETURN. All deal with organizing the exploration of
the program execution into sequential instructions, sometimes involving
some nondeterminism. Consider the simplest case:

\begin{lstlisting}
int main(int argc, char **argv) {
  return foo();
}
\end{lstlisting}

Here, a call to main will be represented as a FUNCTION\_CALL instruction.
The \texttt{code} field contains a \texttt{function\_call} irep that
records what arguments to pass in, the function target (in a \texttt{symbol}
irep), and the variable to assign the return value to. Interpretation begins
in \texttt{goto\_symext::symex\_main}, where we may identify a function as
being an intrinsic, and pass it off to \texttt{goto\_symext::run\_intrinsic}.
All other function calls are passed to
\texttt{goto\_symext::symex\_function\_call} and then
\texttt{goto\_symext::symex\_function\_call\_symbol}.

As part of setup for a function call, we check that the target function
actually exists. Within ESBMC it is valid to call a declared function that has
no body, otherwise we would have to write stubs for every library ever. If no
body exists to be called, a warning is printed to the user, and the return value
of the call becomes nondeterministic. Additionally we check how many times
the target function has been called to detect recursion; if the number of
active function calls exceeds the unwind bound, the function call is ignored.

Following this, if we are still interpreting the function call, we allocate a
new \texttt{framet} object to recall this call, and put it on the top of the
call stack. The tracking data for L1 names is updated so that local variables
get renamed to an L1 name local to this function call, rather than a call to
the same function higher on the stack (\texttt{goto\_symext::locality}).
The arguments to the function are iterated over, and assigned to the relevant
local variables in the called function. A small amount of symex time type
coercion also occurs. Finally we store which instruction to return to, what
variable the return value should go to, and update the program counter to the
first instruction in the target function.

In the course of the functions execution we may run into many RETURN
instructions. Each of these are encoded as an assignment to the return value
in the calling function, stored in the \texttt{framet} object. Following this
we encode an unconditional jump to the end of the function.

Once all paths in the function have been explored we inevitably end up at the
last instruction in the function, possibly with a large number of
\texttt{goto\_statet}'s to be merged in. The final instruction itself is
always a END\_FUNCTION. When this is executed, we reset the program counter to
the calling function and dispose of the \texttt{framet} on the top level of the
stack.

Most of this is straight forward. It becomes more complicated when calls to
function pointers are involved. In the past, the static pointer analysis
has replaced all function pointer calls with code that takes the set of all
functions that may be pointed at, and encoded a mishmash of conditional GOTOs
and function calls. This results in all the functions the pointer may point at
being called, and the correct outcome of the function call being determined
by nondeterministic merging of states (see the interpretation of GOTOs). More
recently, in a slow attempt to remove the static analysis, I've written code
to dereference function pointers at symex time instead. This takes the set of
pointed-to functions from the symex-time pointer tracking, and prepares function
calls to each of them. One is started, then when that function returns the next
function in the list is called. Once there are no more functions to call,
states are merged in the same way as before. There's no massive benefit from
this, but it will permit more control over function call interpretation in
the future. Relevant fields of the \texttt{framet} object are
\texttt{cur\_function\_ptr\_targets}, \texttt{function\_ptr\_call\_loc},
\texttt{function\_ptr\_combine\_target}. and
\texttt{orig\_func\_ptr\_call}.

\subsubsection{Other}

``OTHER'' instructions, as mentioned, are more or less the catch-all instruction
for special cases and ugly hacks. Interpretation of them is redirected to the
\texttt{goto\_symext::symex\_other} method. A general description of what
flavours of OTHER instructions there are lies in the section on GOTO
instructions. Dynamic memory operations are described elsewhere. In an ideal
world we would end up replacing all of these special case instructions with
function call to intrinsics.

\subsubsection{Exception handling}

I know practically nothing about the exception handling (and hopefully others
can write this section). However there are THROW\_DECL instructions that I
believe signify the start of a catch block, with THROW\_DECL\_END signifying
the end of a particular type being caught. There's then the THROW instruction
itself, which probably does the obvious.

\subsection{Multithreaded interactions}

The current way that ESBMC operates is to consider an interleaving wherever
access to a piece of shared state occurs. The context switch itself is defined
to occur \textit{after} the operation that causes it. This doesn't affect
reachability at all so long as there is a context switch available at the start
of any thread. This is because the context switch can actually occur at any
point between two accesses to shared state, it's just most convenient to do
it whenever one of those accesses occurs.

To support this, several \texttt{goto\_symext} methods are overridden in the
\texttt{execution\_statet} class. Then, whenever any assignments, asserts,
assumes, and conditional jumps occur, the normal interpretation is still
applied, but is followed by a call to the \texttt{reachability\_treet} object
that controls the current exploration, to see whether any action should be
taken. What action is actually taken depends on the multithreading exploration
method, see another section.

Two additional instructions are available for interpretation by
\texttt{execution\_statet::symex\_step} --- ATOMIC\_BEGIN and ATOMIC\_END.
These flag the currently executing thread as being in an atomic block,
and not in an atomic block, respectfully. Other threading operations are
available, but implemented in intrinsics rather than specific instructions.
(I'd also prefer atomic begin / end to be intrinsics too).

The \texttt{execution\_statet} class itself can perform many other operations,
reporting facts about the current program state or making operations like
switching which is the current active thread. However these are all called
from \texttt{reachability\_treet}, so I'll describe them elsewhere.

\subsection{Intrinsic library functions}

\section{Dynamic memory}
\section{SMT encoding}
\section{Solving and counterexamples}
\section{Multithreading exploration}
\section{The inevitable 'misc'}
\subsection{Naming and namespaces}
\subsection{Dereferencing}
\subsection{Slicing}
\subsection{irep}
\subsection{Counterexamples}
\section{Conclusion}
\end{document}
