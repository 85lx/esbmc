%\documentclass[oribibl,runningheads,a4paper]{llncs}
\documentclass[a4paper]{llncs}
\usepackage{url}
\usepackage{times}
\usepackage{listings}
\usepackage{graphicx,marvosym}

\newcommand{\comment}[1]{}
\newcommand{\citenot}[1]{}
\newcommand{\blurb}[1]{{\texttt{[.. #1..]}}}


\begin{document}

\lstset{language=C,basicstyle=\small}
\lstset{numbers=left, numberstyle=\tiny, stepnumber=1, numbersep=5pt}
\lstset{tabsize=2}
\lstset{firstnumber=1}
\lstset{frame=single}
\lstset{
  language={C},
  morekeywords={assert,uchar}
}

%\mainmatter  % start of an individual contribution

%\title{Context-Bounded Symbolic Model Checking with ESBMC 1.17}
\title{SMT-Based Bounded Model Checking of C++ Programs}
%\subtitle{(Competition Contribution)}
%\title{TACAS'12 Competition Entry: ESBMC v1.17}
%\title{Software Verification Competition: ESBMC v1.17}
%\title{TACAS'12 Software Verification Competition: ESBMC v1.17}
%\title{System Description: ESBMC v1.17}
%\title{System Description: ESBMC v1.17 (TACAS'12 Competition version)}
%\titlerunning{}
\comment{
\author{Lucas Cordeiro$^1$ \and
	Jeremy Morse$^2$   \and
	Denis Nicole$^2$   \and
	Bernd Fischer$^2$}
\authorrunning{Lucas Cordeiro, Jeremy Morse, Denis Nicole, Bernd Fischer}
\institute{
  $^1$ Electronic and Information Research Center, %\\
  Federal University of Amazonas, Brazil\\
  %\url{lucascordeiro@ufam.edu.br}
  %\\\smallskip
  $^2$ Electronics and Computer Science, %\\
  University of Southampton, UK\\
  %\url{{jcmm106,dan,bf}@ecs.soton.ac.uk}
  %\url{http://www.ecs.soton.ac.uk}
  \url{esbmc@ecs.soton.ac.uk}
}
}

\maketitle

\begin{abstract}
Bounded model Checking (BMC) of C++ programs presents greater complexity than C programs due to features that the languages offers, such as templates, containers and exception handling. We present ESBMC, a bounded model checker for C and C++ programs, which encodes the verification conditions using different background theories and passes them directly to an SMT Solver. Our experimental results show that our approach can handle more constructs of the C++ programs and substantially reduce the verification time. 
\end{abstract}

\section{Introduction}
%

Bounded model checking (BMC) has already been successfully applied to
verify software and to discover subtle errors in real
systems~\cite{handbook09}.\ In an attempt to cope with growing 
system complexity, Boolean Satisfiability (SAT) solvers are increasingly 
replaced by Satisfiability Modulo Theories (SMT) solvers to prove the generated 
verification conditions (VCs) ~\cite{Armando09,Ganai06,Cordeiro12}.\
There have also been attempts to extend BMC to the verification of C++ programs.
The main challenge here is to support the features that the languages offers, 
such as templates, containers and exception handling.

There are several real-world applications that use C++ as its mainstream programming 
language, which includes software/hardware verification tools, information retrieval 
machines, databases, simulators, embedded systems, and telecommunication systems. 
If we compared the C++ to the C programming language, C++ arises much more challenges 
since it provides a wider set of features, libraries and functionalities that would require 
too much effort to develop from scratch. These features include Object-Oriented Programming (OOP), 
specialized input-output libraries (e.g., stream libraries), and the template usage 
(including STL containers, which have the most popular data structures in Computing Science).

Systems that make use of the C++ language (and its functionalities) tend to require a high 
verification effort since the errors are hard to find as its own structure grows. 
To hunt bugs in C++ applications, software verification tools are now becoming more and more important 
and critical due to its crucial characteristics (e.g., speed, accuracy, efficiency, and friendliness). 
Obviously, if the C++ programming language seems to be a more complex version than C, then its verification 
will be more complex as well. To tackle this problem, our solution applies Bounded Model Checking (BMC) 
to C++ programs using an operational model of the C++ libraries. In particular, we focus on the STL 
sequential containers operational model, its preconditions and simulation features 
(e.g., how we store the elements values of the containers and intern class methods), 
and how this is used to verify the C++ program.

\begin{figure*}[ht]
\centering
\includegraphics[scale=0.28]{figures/diagramas-cpp}
%\includegraphics[scale=0.28]{figures/SystemArchitecture-eps-converted-to.pdf}
\caption{Representation of the configuration and functional classification of the operational model.}
\label{figure:cpp-diagram}
\end{figure*}

\section{Background}
%
Background here (Lucas)...

\section{C++ Operational Model}
%
During the verification process, ESBMC needs to identify all the features and specifications of the code in order to generate the AST. In case of the C++ code, the basic representations (e.g., features related to object orientation, template and exception handling) are set internally to ESBMC in different levels (typecheck, goto-program, and goto-symex). However, ESBMC must have the specifications related to definitions of C++ libraries such as classes, methods, and types. Keeping this in mind, we developed a simplified representation of C++ libraries, with their respective classes, methods, and other features, called C++ Operational Model (COM).


The development process of the COM can be divided into two phases: structural and modeling. In the structural phase, we built a set of classes with a specific hierarchical relationship and the signature of their methods and specific data, which consequently resulted in a simplified structure representing the set of C++ libraries. From this structure, we modeled the methods of each class and this modeling is focused for the validation and verification of all properties that include the specific method.
%
\subsection{Structure and Model}
%
In general, the verification process of a program ESBMC can be divided into two segments, respectively, identifying the expressions and verifying all properties related to those expressions, and in the identification step, the ESBMC must support all syntactic features of the language in which the program is implemented. Based on this, we have adopted two approaches to the development of the support of all syntactic characteristics of a C++ code. The first is to identify the basic structures of language, for example, polymorphism and template, modifying the kernel of the tool so that the AST can contain all the necessary information to verify these structures. The second approach is directly related to the definitions (classes, methods, functions, types) present in the C++ libraries, the goal is to build an operational model so that the ESBMC can rely on it to identify such definitions and also, verify all the properties related to these definitions based on the implementation of this model. It is important to say that this model is inserted into the verification process at the level of source code, ie, both model and source code are passed as parameters at the beginning of the verification process, so that the scan, parser and typecheck are done with the purpose of creating the AST joining the implementations.

The first step of the construction process of this operating model is the formation of a simplified structure, which should get closer to actual structure as possible. This way, based on the documentation of the language C++ [1] the set of libraries was divided according to their functionalities, into four subsets called C Libraries, Input/Output Stream, Standard Template Libraries, and General Libraries, respectively. Each library present in a subset is particularly related to the other libraries, not only in its functionality but also in its structure due to the fact that many libraries depend on definitions of others. From this, we performed an analysis in the respective subsets, Figure 1, such a way that could be identified dependencies between each library and thus develop a simplified structure of each. The topics that follow describe the structure of subsets of libraries shown in Figure 1.
%
\subsection{C Libraries}
%
The C++ Standard Libraries also includes all the ANSI-C libraries, seeing that it's an extension of this language. The ESBMC supported language ANSI-C, however when occurred verification in C++ code; the same follow one different way for kernel's tool, passing for a specific typecheck directed to C++ software. For this reason, it's also necessary to build a representation of the ANSI-C libraries' set in the operational model. Meantime, we can define this libraries' set in simplified structure consisting of macro definitions and functions. Among the ANSI-C libraries' set exist libraries containing only macro definitions, for example, the ciso646 library that define a spelling set to the logic operators. To build the structure of this library, we distingue two kind of spelling: we define the macro set $M$ and operators set $O$. For convention, we assume $\left\{and, \: and\_eq, \: bitand, \: bitor, \: compl, \: not, \: not\_eq, \: or, \: or\_eq, \: xor, \: xor\_eq\right\} \subset M$, and $\left\{\&\&, \: \&=, \: \&, \: |, \: \widetilde{} \:, \: !, \: ||, \: |=, \: \widehat{} \:, \: \widehat{}= \: \right\} \subset O$. Therefore, we define the syntax for these definitions set as:
\\\\
Within the ANSI-C libraries set, also we can represent the structure of others libraries with functions only. This is case of the csignal library that dealing with signals emitted in a given code. Basing in this concept, we define that the operational model of this library has been represented to signal and raise functions, respectively, and furthermore, to vector $v_{i}$ assuming that $\left\{ i \in \aleph \:|\: 0 \leq i \leq 15 \right\}$ and to finite signals set $A$ assuming that $\left\{2, 4, 6, 8, 11, 15\right\}\:\subset\:A$. The functional signal records certain function in one specific signal such that when the function raise is called with the respective signal the registered function is called, therefore, assuming f as a un-deterministic function and as null element, is obtained following the formalize modeling.
\\
\\
The code section that represents this structure can be observed in the following.
%

\begin{figure}[ht]
\centering
\begin{minipage}{0.7\textwidth}
\begin{lstlisting}
#ifndef STL_CSIGNAL
#define STL_CSIGNAL
typedef void (*sighandler)(int);
sighandler vetor[16];
  void signal(int sig, sighandler func);
  int raise(int sig);
#endif
\end{lstlisting}
\end{minipage}
\caption{Structure of the CSIGNAL.}
\label{figure:structure-of-the-CSIGNAL}
\end{figure}


\subsection{Input / Output Stream Libraries}
The libraries group related to flow control of input and output data inside to certain program is represented for the libraries' set called Input / Output Stream. Aiming to build the operational model of this set, we elaborated a hierarchical structure really similar to real that can be observed in Figure 4.
\subsection{Standard Template Libraries}

This set is subdivided in four categories: Containers, Algorithms, Iterators and Numeric. The algorithm library is the only one that composes the Algorithms categories and it has a model formed by 66 functions that present as main objective the manipulation of STL containers. Therefore, we define the variables set $It$ representing ranges and $Val$ representing any values, we assume $\left\{it_{1}, it_{2}\right\} \:\subset\:It$ and $\left\{n\right\} \:\subset\:Val$. From this, we can exemplified the functionalities of this library through the model's find function that search between the range compose to $it_{1}$, pointed to the first element of range, and $it_{2}$, pointed to the last element of range, the value represented to $n$. The modeling of this function has the following formal representation:
\\
	The Numeric category present the numeric library formed by four functions that manipulated numeric sequences basically. To exemplified the functionalities of this library, we take as exempla the accumulate function; it covers a certain range adding the values inside the same in one variable (accumulator). Keep this in mind, we assume $\left\{it_{1}, it_{2}\right\} \:\subset\:It$ and $\left\{n\right\} \:\subset\:Val$, obtaining the modeling following:

\subsubsection{Structure.}

The structure of STL containers is based on the C++ structure itself, including its classes, operators, methods, functions and intern variables. It is divided in: iterations, capacity, element access, modifiers and unique members. The similarities and differences between sequential containers are described at Table 1.

%%Hendrio: you should summarize the table in the text and then cite it.
\comment{	
\begin{table}
	\centering
		\begin{tabular}
			
		\end{tabular}
	\caption{STL sequential containers}
	\label{tab:STLSequentialContainers}
\end{table}
}
	
\subsubsection{Model Semantics.}
	
	Let us consider that a container model is composed by five types of variables, \emph{I}, \emph{C}, \emph{N}, \emph{P} and \emph{T}. \emph{I} represents a iterator that points to a position in the container, \emph{C} represents the container itself, N represents natural integer numbers used in the container, like size, capacity and elements index, \emph{P} represents the memory address where \emph{T} is located, and \emph{T} represents de values stored in the container. For convention, we assume that \emph{\{c, v, d, l\}} $\subset C, \{i, j, n\} \subset N$ and $\{it1, it2\} \subset I$. 
	
	
	The containers contain an array of elements \emph{T}, and their positions in the memory are represented by pointers \emph{P}.
Assuming this, the syntax for integer expression is:
%
\[\begin{array}{r@{\:\:}c@{\:\:}l}
\\[-5ex]
\mathit{Int}  & ::= & \: \mathit{N} \: | \: \mathit{Z} \: | \: \mathit{C.size} \: | \: \mathit{C.capacity} \: | \\
              &     & \: \mathit{Int} ( + \: | \: ? \: | \: * \: | \: ...) \mathit{Int}  \: | \\
              &     & \: \textit{It} ( + \: | \: - ) \textit{It} 
\end{array}
\]
%
Similarly, the syntax for iterator expressions is:
%
%
\[\begin{array}{r@{\:\:}c@{\:\:}l}
\\[-5ex]
\mathit{It}   & ::= & \: \mathit{I} \: | \: \mathit{It} ( + \: | \: - ) \mathit{It} \: | \: \mathit{C.begin} \: | \: \mathit{C.end} \: | \\
\end{array}
\]
%


	For \emph{P} (memory address values), the syntax is as follows:
%	\emph{
%		\textit{P} = p \left| \textit{It}.pointer \left| \textit{C}.array
%		\left| \textit{It}.source
%		\left| \textit{P} ( + \left| - ) \textit{P}
%	}
%\\ \\
%	The syntax for \emph{T} values is the following: 
%\\ \\
%\emph{	
%	\textit{T} = t \left| *\textit{It} \left| *\textit{P} \left| \textit{C}_int
%	}
%	\\ \\
%	To test the assertions, we use Booleans expressions, with the following syntax: 
%	\\ \\
%		Bool = Int ( < | > | = | � ) Int 
%| It ( < | > | = | � ) It
%| T  =  T
% | Assert | Assert (_| ^| ...) Assert
%|    var . Assert | E var . Assert

\subsubsection{Model.}
	To simulate appropriately the containers, our model makes use of three variables: a variable \textit{P} called array, that points to the first element of the elements array, a natural value size, that stores the quantity of elements contained in the container, and a natural value capacity, that stores the total capacity of a container (valid onliy for vectors). Note that, as the elements are added in the container (specifically in vectors) and the size grows, the capacity also grows at a rate of 2*size, every time the size reaches the capacity value.
	Similarly, iterators are modeled using three representing variables: a variable P called pointer, which contains the memory address to the correspondent element \textit{T} in the container, a variable N called position, which contains the index value pointed by the iterator, in the container, and a variable P called source, which contains the memory address correspondent to the first element \textit{T} stored in the container.
	The vector container model has a structure as it follows:
	
\emph{	C1 = \{P1, C1.size, C1.capacity\}}

	Where P1 is a memory address where it is stored the elements of the container, C1.size is the total number of elements in the container, and C1.capacity is the total capacity of the vector, simulated internally in the model. \\ \\
	The main methods of a vector (and sequential containers, in general) have only three types of operation: insertion, exclusion and search. Methods like push-back(), pop-back(), front(), back(), push-front() and pop-front() are only a simplified variation of those main methods, optimized for some containers (like pop-back in a stack).

	An insert method is represented by the following structure:

	C{cont}
	It{cont.insert}|cont.insert
	It{position} \& It{first} \& It{last}
	P{point1} \& P{point2}
	T{value}
	N{quantity}
	cont.insert (position, value)
		cont'.size = cont.size + 1
		*position = value
	|cont.insert(position, value, quantity)
		cont'.size = cont.size + quantity
		*(position + N(0 -> quantity)) = value
	|insert(position, first, last)
		cont'.size = cont.size +( last - first)
		*position = *first
	|insert(position, point1, point2)
		cont'size = cont.size + (point2 - point1)
		*position = *point1

	Similarly, an erase method is structured like the following:

	C{cont}
	It{cont.erase}|cont. erase
	It{position} \& It{first} \& It{last}
	P{point1} \& P{point2}
	cont. erase (position)
		cont'.size = cont.size - 1
		position' = position + 1
	|cont. erase (position, first, last)
		cont'.size = cont.size -(last - first)
		position' = last

	Searches are made in a container by using reference operators and a pointing type (pointer or iterator), and return the reference value (the element stored itself).

	C{cont}
	It{i}
	N{n}
	T{C[n]}
	T{*i}

	The structure of iterators is treated differently from the other types. The model is the following:
	
	It : P{pointer}; N{pos}; P{cont\_pos}

	Where pointer is a memory address that points to the real position of the required element in the container (pointed by the iterator), pos is the iterator index internally in the container, and cont\_pos is a memory address equivalent to cont.buf, being cont the container pointed by the iterator.

%
\subsection{General Libraries}


\section{Experimental Results}
%
Experimental Results here (Mauro)...



\section{Related Work}
%
Related Work here (Mauro and Mikhail)

\section{Conclusions}
%
Conclusions here (Lucas)...


\smallskip{\small\noindent{\bf Acknowledgments.} 
%
Acknowledgments here (Lucas)...
}

\vspace{-2.5ex}
\renewcommand\refname{{\normalsize References}}
{\begin{thebibliography}{10}
\vspace{-0.5ex}

%\bibitem{Boolector09}
%R.~Brummayer and A.~Biere
%\newblock Boolector: An efficient {SMT} solver for bit-vectors and arrays
%\newblock {\em TACAS}, {\em LNCS} 5505, pp. 174--177, 2009.

\bibitem{handbook09}
A.~Biere.
\newblock Bounded model checking.
\newblock In {\em Handbook of Satisfiability}, pp. 457--481. 2009.

\bibitem{Cimatti10}
A.~Cimatti, A.~Micheli, I.~Narasamdya, and M.~Roveri.
\newblock Verifying {SystemC}: a software model checking approach.
\newblock {\em FMCAD}, 2010, pp.\ 121--128.

\bibitem{Clarke04}
%E.~Clarke et~al.
E.~Clarke, D.~Kroening, and F.~Lerda.
\newblock A tool for checking {ANSI-C} programs.
\newblock {\em TACAS}, {\em LNCS} 2988, pp.\ 168--176, 2004.

\bibitem{CordeiroPhD}
L.~Cordeiro.
\newblock {SMT}-Based Bounded Model Checking of Multi-Threaded Software in 
Embedded Systems.
\newblock PhD Thesis, U Southampton, 2011.

\bibitem{Cordeiro09}
L.~Cordeiro, B.~Fischer, and J.~Marques-Silva.
\newblock {SMT}-based bounded model checking for embedded {ANSI-C} software.
\newblock {\em ASE}, pp.\ 137--148, 2009. 
%\newblock Extended version to appear in \emph{IEEE Trans.~Software Engineering}.

\bibitem{icse11}
L.~Cordeiro and B.~Fischer.
\newblock Verifying Multi-Threaded Software using {SMT}-based Context-Bounded 
Model Checking.
\newblock {\em ICSE}, pp.\ 331--340, 2011. 

%\bibitem{Z08}
%L.~M. de~Moura and N.~Bj{\o}rner.
%\newblock Z3: An efficient {SMT} solver.
%\newblock {\em TACAS}, {\em LNCS} 4963, pp. 337--340, 2008.

\bibitem{sefm11}
J.~Morse, L.~Cordeiro, D.~Nicole, and B.~Fischer.
\newblock Context-Bounded Model Checking of LTL Properties for ANSI-C Software.
\newblock {\em SEFM}, {\em LNCS} 7041, pp.\ 302--317, 2011.

\bibitem{Armando09}
A.~Armando, J.~Mantovani, and L.~Platania.
\newblock Bounded model checking of software using {SMT} solvers instead of
  {SAT} solvers.
\newblock {\em STTT}, vol. 11 (1), pp. 69--83, 2009.

\bibitem{Ganai06}
M.~K. Ganai and A.~Gupta.
\newblock Accelerating high-level bounded model checking.
\newblock {\em ICCAD}, pp. 794--801, 2006.

\bibitem{Cordeiro12}
L.~Cordeiro, B.~Fischer, and J.~Marques-Silva.
\newblock {SMT}-based bounded model checking for embedded {ANSI-C} software.
\newblock {\em IEEE Trans. Software Eng.}, v.\ 38, n.\ 4, pp.\ 957--974, 2012. 
%\newblock Extended version to appear in \emph{IEEE Trans.~Software Engineering}.

%\bibitem{wtr}
%R.~Barreto, L.~Cordeiro, and B.~Fischer.
%\newblock Verifying Embedded C Software with Timing Constraints using an
%Untimed Bounded Model Checker
%\newblock {\em Proc.\ SBESC Workshop on Real-Time Systems}, to appear, 2011.

%\bibitem{Godefroid95}
%Patrice Godefroid.
%\newblock {\em Partial-order Methods for the Verification of Concurrent
%  Systems: An Approach to the State-explosion Problem}.
%\newblock University of Liege, PhD thesis, 1995.

\end{thebibliography}}

\end{document}

